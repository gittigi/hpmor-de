

\hypertarget{humanismus-teil-2}{% \section{12. Humanismus, Teil 2}\label{humanismus-teil-2}}

—\/-\/-\/-\/- Kapitel 44: Humanismus, Teil 2 -\/-\/-\/-\/-

„Fawkes“, sagte Albus Dumbledore, seine Stimme brach, „hilf ihm, bitte—“

Eine schimmernde Kreatur aus Rotgold schob sich ins Blickfeld, schaute fragend nach unten; und sie begann zu singen.

Das bedeutungslose Zwitschern glitt von der Leere ab, es gab nichts, woran es sich festhalten konnte.

„Du bist laut“, sagte die Stimme, „du solltest sterben.“

„Schokolade“, sagte Albus Dumbledore, „du brauchst Schokolade, und deine Freunde - aber ich wage nicht, dich zurückzubringen—“

Dann kam ein strahlender Rabe und sprach mit Professor Flitwicks Stimme, woraufhin Albus Dumbledore in plötzlicher Einsicht keuchte und über seine eigene Dummheit laut fluchte.

Das leere Ding lachte darüber, denn es hatte sich die Fähigkeit bewahrt, sich zu amüsieren.

Und einen Augenblick später waren sie alle in einem weiteren Feuerblitz verschwunden.

—\/-\/-\/-\/-\/-\/-\/-\/-\/-\/-\/-\/-\/-\/-\/-\/-\/-\/-\/-\/-\/-\/-\/-\/-\/-\/-\/-\/-\/-\/-\/-\/-\/-\/-\/-\/-\/-\/-\/-\/-\/-\/-\/-\/-\/-\/-\/-\/-\/-\/-\/-\/-\/-\/-

Es war nur ein Moment, so schien es, zwischen dem Moment, in dem Flitwicks Rabe an einen anderen Ort geflogen war, und dem Moment, in dem Albus Dumbledore mit Harry in seinen Armen in einer weiteren Spalte aus rot-goldenem Feuer wieder auftauchte; aber irgendwie hatte es Hermine in dieser Zeit schon geschafft, ihre Hände mit Schokolade zu füllen.

Bevor Hermine überhaupt dort ankam, war die Schokolade vom Tisch und direkt in Harrys Mund geflogen, was ein winziger Teil ihres Verstandes als unfair empfand, \emph{er} hatte die Chance bekommen, es für \emph{sie} zu tun—

Harry spuckte die Schokolade wieder aus.

„Geh weg“, sagte eine Stimme, die so leer war, dass sie nicht mal kalt war.

…

Alles schien zu erstarren, jeder, der sich auf Harry zubewegt hatte, blieb stehen, alle Bewegungen wurden durch den Schock dieser beiden toten Worte unterbrochen.

Dann: „Nein“, sagte Albus Dumbledore, „das werde ich nicht“, und die Zeit nahm wieder ihren Lauf, sogar ein weiteres Stück Schokolade zischte vom Tisch in Harrys Mund.

Hermine war jetzt nahe genug, um zu sehen, wie Harrys Gesichtsausdruck noch hasserfüllter wurde, als sein Mund mit einem mechanischen, unnatürlichen Rhythmus kaute.

Die Stimme des Schulleiters war stählern. „Filius, ruf Minerva, sag ihr, sie muss schnell herkommen.“

Professor Flitwick flüsterte seinem silbernen Raben zu, der in die Luft flog und verschwand.

Ein weiteres Stück Schokolade schwebte in Harrys Mund, und das mechanische Kauen ging weiter.

Es versammelten sich noch mehr Schüler dort wo der Schulleiter mit grimmigen Augen über Harry wachte: Neville, Seamus, Dean, Lavender, Ernie, Terry, Anthony, keiner von ihnen wagte es, näher heranzukommen als Hermine.

„Was können wir tun?“, sagte Dean mit zitternder Stimme.

„Geht zurück und gebt ihm mehr Platz—“ sagte die trockene Stimme von Professor Quirrell.

„Nein!“ unterbrach der Schulleiter. „Lasst ihn von seinen Freunden umgeben sein.“

Harry schluckte seine Schokolade und sagte mit dieser leeren Stimme: „Die sind dumm. Sie sollten \emph{sterbenmmmpfffff}“, als ein weiteres Stück Schokolade in seinen Mund eindrang.

Hermine sah die schockierenden Blicke, die ihre Gesichter durchzogen.

„Er meint das nicht so, oder?“ Seamus sagte es, als würde er betteln.

„Du verstehst das nicht“, sagte Hermine, ihre Stimme brach, „\emph{das ist nicht Harry}—“ und sie hielt den Mund, bevor sie noch mehr sagte, aber so viel \emph{musste} sie sagen.

Sie sah an seinem Gesichtsausdruck, dass Neville es verstand, und sie sah auch, dass die anderen es nicht taten. Wenn Harry wirklich nie so etwas gedacht hätte, dann hätte ihn die Tatsache, dass er weniger als eine Minute einem Dementor ausgesetzt war, nicht dazu gebracht, es zu sagen. Das haben sie wahrscheinlich gedacht.

Weniger als eine Minute Dementor-Exposition könnte nicht aus dem Nichts einen neuen bösen Menschen in dir schaffen.

Aber wenn diese Person \emph{schon da} wäre…

\emph{Weiß der Schulleiter davon}?

Hermine schaute zum Schulleiter auf und stellte fest, dass Albus Dumbledore \emph{sie} anstarrte und dass seine blauen Augen plötzlich stechend geworden waren—

Ihr kamen Worte in den Sinn.

\emph{Sprich nicht darüber}, sagte der Wille von Dumbledore zu ihr.

\emph{Sie wissen}, dachte Hermine. \emph{Über seine dunkle Seite}.

\emph{Ich weiß. Aber das hier geht darüber hinaus. Fawkes' Lied kann ihn dort nicht erreichen, wohin er verschwunden ist.}

\emph{Was können wir}…

\emph{Ich habe einen Plan}, schickte den Direktor. \emph{Geduld}.

Irgendwas am Tenor dieses Gedankens machte Hermine nervös. \emph{Was für ein Plan}?

\emph{Es ist besser, wenn du es nicht weißt,} sendete der Schulleiter.

Jetzt wurde Hermine \emph{richtig} nervös. Sie wusste nicht, wie \emph{viel} der Direktor über Harrys dunkle Seite wusste.

\emph{Ein guter Punkt}, schickte der Direktor. \emph{Ich bin dabei, es dir zu sagen; stähle dich, um nicht zu reagieren. Bist du bereit? Ja, ich bin bereit. Ich werde so tun, als würde ich den tödlichen Fluch auf Professor} \emph{McGonagall} \emph{werfen. REAGIERE NICHT Hermine!}

Das kostete Arbeit. Der Schulleiter war wirklich verrückt! Das würde Harry nicht aus seiner dunklen Seite reißen, er würde \emph{völlig durchdrehen}, er würde den Direktor \emph{töten}.

\emph{Aber das ist nicht die wahre Dunkelheit}, schickte Albus Dumbledore. \emph{Das ist Beschützerverhalten, das ist Liebe}. \emph{Dann wird Fawkes ihn erreichen können. Und wenn Harry sieht, dass Minerva doch noch lebt, wird er ihn vollends zurückbringen.}

Der Gedanke kam zu Hermine—

\emph{Ich bezweifle, dass das funktionieren wird,} schickte der Schulleiter, \emph{und es könnte Ihnen nicht gefallen, wie er reagiert, wenn Sie es versuchen. Aber Sie können es versuchen, wenn Sie wollen.}

Sie hatte das nicht wirklich ernst gemeint! Es war zu…

Dann bewegten sich ihre Augen, brachen den Blickkontakt mit dem Schulleiter, glitten zu dem Jungen, der mit leeren, verachtenden Augen um sich schaute, während sein Mund weiter kaute und Tafel um Tafel Schokolade ohne Effekt schluckte.

Ihr Herz zerriss, und plötzlich schien vieles nicht mehr wichtig zu sein, nur dass es eine Chance gab.

—\/-\/-\/-\/-\/-\/-\/-\/-\/-\/-\/-\/-\/-\/-\/-\/-\/-\/-\/-\/-\/-\/-\/-\/-\/-\/-\/-\/-\/-\/-\/-\/-\/-\/-\/-\/-\/-\/-\/-\/-\/-\/-\/-\/-\/-\/-\/-\/-\/-

Es gab den Zwang, Schokolade zu kauen und zu schlucken. Die Antwort auf Zwang war Töten.

Die Menschen hatten sich versammelt und starrten. Das war ärgerlich. Die Reaktion auf Verärgerung war Töten.

Andere Leute plapperten im Hintergrund. Das war unverschämt. Die Antwort auf Unverschämtheit war, Schmerzen zuzufügen, aber da keiner von ihnen nützlich war, wäre es einfacher, sie zu töten.

All diese Menschen zu töten, wäre schwierig. Aber viele von ihnen trauten Quirrell nicht, der stark war. Den richtigen Auslöser zu finden, könnte dazu führen, dass sie sich alle gegenseitig umbrachten.

Dann beugte sich eine Person in das Blickfeld und tat etwas völlig Seltsames, etwas, das zu einer fremden Denkweise gehörte, für die es nur eine einzige, irgendwo gespeicherte Reaktion gab—

—\/-\/-\/-\/-\/-\/-\/-\/-\/-\/-\/-\/-\/-\/-\/-\/-\/-\/-\/-\/-\/-\/-\/-\/-\/-\/-\/-\/-\/-\/-\/-\/-\/-\/-\/-\/-\/-\/-\/-\/-\/-\/-\/-\/-\/-\/-\/-\/-

Sie hörte das Keuchen um sie herum, und das machte nichts, sie hielt den Kuss auf diesen schokoladenverschmierten Lippen aufrecht, während die Tränen aus ihren Augen quollen.

Und Harrys Arme kamen hoch und stießen sie weg, und seine Lippen schrieen: „\emph{Ich habe dir gesagt, kein Küssen!}“

—\/-\/-\/-\/-\/-\/-\/-\/-\/-\/-\/-\/-\/-\/-\/-\/-\/-\/-\/-\/-\/-\/-\/-\/-\/-\/-\/-\/-\/-\/-\/-\/-\/-\/-\/-\/-\/-\/-\/-\/-\/-\/-\/-\/-\/-\/-\/-\/-

„Ich glaube, er wird jetzt wieder gesund“, sagte der Schulleiter und sah, wie Harry in großen, jämmerlichen Schluchzern weinte, als Fawkes über ihm sang. „Hervorragend gemacht, Miss~Granger. Wissen Sie, nicht einmal ich hätte erwartet, dass das wirklich funktioniert?“

Das Lied des Phönix war nicht für sie bestimmt, wusste Hermine, aber sie konnte sich trotzdem davon besänftigen lassen, was sie brauchte, denn ihr Leben war offiziell vorbei.

