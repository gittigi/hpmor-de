

\hypertarget{glauben-in-glauben}{% \section{2. Glauben in Glauben}\label{glauben-in-glauben}}

Kapitel 2\\ Glauben in Glauben

„Und dann war da Janet, eine Squib,“ sagte das Portrait einer kleinen, jungen Dame mit einem goldbesetzten Hut.

Draco schrieb es auf. Es waren nur achtundzwanzig, aber es war Zeit zurückzukehren und Harry zu treffen.

Er hatte andere Portraits um Hilfe bei der Übersetzung bitten müssen, Englisch hatte sich sehr verändert, doch auch die ältesten Portraits hatten stets nur Erstklässler-Zauber beschrieben, die sehr ähnlich wie die Heutigen klangen. Draco hatte etwa die Hälfte erkannt und die Restlichen klangen nicht stärker.

Das mulmige Gefühl in seinem Magen hatte sich mit jeder Antwort weiter ausgebreitet, bis er schließlich es nicht mehr ausgehalten hatte und stattdessen andere Portraits Harrys merkwürdige Frage über Squibehepaare gefragt hatte. Die ersten fünf Portraits hatten keines gekannt und zum Schluss hatte er diese Portraits gebeten, dass sie \emph{ihre Bekannten darum bitten sollen, wiederum \emph{ihre} Bekannten zu fragen und somit es erreicht, einige Leute zu finden, die wirklich zugaben mit Squibs befreundet zu sein.}

(Der Slytherin des ersten Jahres hatte erklärt, dass er zusammen mit einem Ravenclaw an einem wichtigen Projekt arbeite und dass der Ravenclaw ihm mitgeteilt hatte, dass die Information nötig wäre und dann aber einfach davon gegangen war ohne sich zu erklären. Das hatte ihm einige mitfühlende Blicke eingebracht.)

Dracos Füße waren schwer als er durch die Korridore von Hogwarts ging. Er sollte rennen, aber er schien nicht in der Lage zu sein diese Energie auf zu bringen. Ihm schwierte im Kopf herum, dass er nichts davon wissen wollte, dass er nicht darin verwickelt sein wollte, dass es nicht in seiner Verantwortung lag, dass Harry Potter es machen sollte, dass wenn die Magie verblasste Harry sich darum kümmern könnte…

Aber Draco wusste, dass es nicht richtig war.

Kalt war es in den Kerkern von Slytherin, grau waren die Steinwände, Draco mochte normalerweise diese Atmosphäre, aber jetzt wirkte es zu sehr wie Verblassen.

Seine Hand auf dem Türknauf, Harry Potter bereits drinnen auf ihn wartend, seinen Kapuzenumhang tragend.

„Die uralten Erstklässler-Zauber,“ sagte Harry Potter. Was hast du herausgefunden?“

„Sie sind nicht mächtiger als die Zauber, die wir jetzt verwenden.“

Harry Potters Faust schlug auf den Tisch, hart. „Verdammt. Ok. Mein eigenes Experiment war ein Fehlschlag, Draco. Es gibt soetwas namens das Interdikt von Merlin ----“

Draco schlug sich die Hand vor die Stirn, er verstand es.

„ ---- das jeden daran hindert Wissen über mächtige Zauber aus Büchern zu gewinnen, selbst wenn du die Notizen eines mächtigen Zauberers findest und liest, wirst du sie nicht verstehen können, es muss von einem lebenden Verstand zum Verstand weitergegeben werden. Ich konnte keine Zauber finden, für die wir Anleitungen haben, aber nicht verwenden können. Doch wenn man sie nicht aus alten Büchern lernen kann, warum sollte jemand sie mündlich weitergeben, nachdem sie aufgehört haben zu funktionieren? Hast du Daten zu Squibpaaren bekommen?“

Draco machte Anstalten das Pergament herüberzureichen ----

Aber Harry Potter hob eine Hand. „Gesetz der Wissenschaft, Draco. Zu Beginn erzähl ich dir die Theorie und die Vorhersage. Dann zeigst du mir die Daten. Damit du weißst, dass ich mir nicht nur eine passende Theorie ausdenke; du weißst dann, dass die Theorie wirklich die Daten im \emph{Voraus vorhergesagt hat. Ich muss dir das in jedem Fall erklären, also muss ich es dir erklären \emph{bevor} du mir die Daten zeigst. Das ist die Regel. Also zieh deinen Umhang an und lass uns anfangen.“}

Harry Potter setzte sich an einen Tisch, auf dem zerrissene Stücke Papier verteilt waren. Draco nahm seinen Umhang aus seiner Büchertasche, zog ihn an, und setzte sich, den Papierfetzen einen verwirrten Blick zuwerfend, gegenüber von Harry auf der anderen Seite des Tisches hin. Sie waren in zwei Reihen aufgebaut mit den Reihen etwa zwanzig Stücke lang.

„Das Geheimnis des Blutes,“ sagte Harry Potter, mit einem angestengten Blick auf seinem Gesicht, „ist etwas namens Desoxyribonukleinsäure. Diesen Namen verwendest du nicht vor jemandem, der kein Wissenschaftler ist. Desoxyribonukleinsäure ist das Rezept nach dem dein Körper wächst, zwei Beine, zwei Arme, klein oder groß, ob du braune Augen oder grüne hast. Es ist etwas Materielles, du kannst es sehen, wenn du Mikroskope verwendest - die sind wie Teleskope nur das sie Dinge zeigen, die sehr klein sind, statt sehr weit weh. Und dieses Rezept hat zwei Kopien von allem, immer, falls eines mal kaputt geht. Stell dir zwei lange Reihen von Papierfetzen vor. An jedem Platz in der Reihe, gibt es zwei Stücke Papier und wenn man Kinder hat, wählt der Körper zufällig ein Stück an jedem Platz der Reihe aus und der Körper des Partners macht das Gleiche und so hat das Kind auch zwei Stücke Papier in der Reihe. Zwei Kopien von allem, eines von der Mutter, eines vom Vater und wenn das Kind dann wieder Kinder hat, erhalten sie zufällig ein Stück Papier von ihm an jedem Platz.“

Während Harry sprach, zeigte er mit seinen Fingern auf die paarweise hingelegten Papierfetzen, auf ein Teil eines Paares deutend als er „von der Mutter“, auf das Andere als er „vom Vater“ sagte. Und als Harry darüber sprach zufällig auszuwählen, nahm seine Hand einen Knut hervor und warf ihn; Harry sah auf die Münze und deutete dann auf das obere Stück Papier. Alles ohne eine Pause in seiner Rede zu machen.

„Wenn es jetzt wie Eigenschaften um wie groß oder klein zu sein geht, gibt es \emph{sehr} viele Plätze im Rezept, die \emph{kleine} Unterschiede machen. Sodass wenn eine großer Vater eine kleine Mutter heiratet, das Kind einige Papierstücke erhält die „groß“ sagen und einige Papierstücke die „klein“ sagen und üblicherweise wird das Kind durchschnittlich groß. Aber nicht immer. Mit Glück, könnte das Kind viele Stücke, die „groß“ sagen und nicht viele die „klein“ sagen erhalten und ziemlich groß werden. Man könnte auch einen großen Vater mit 5 Papieren, die „groß“ sagen und eine große Mutter mit ebenfalls 5 Papieren, die „groß“ sagen haben und mit herausragendem Glück erhält das Kind alle zehn Papiere, die„groß“ sagen und wird schließlich größer als beide. Verstehst du? Blut ist keine perfekte Flüssigkeit, es vermischt sich nicht vollkommen. Desoxyribonukleinsäure besteht aus einigen aus lauter kleinen Stücken, wie ein Glas von Kieselsteinen, statt wie ein Glas Wasser. Das is der Grund warum ein Kind nicht immer genau in der Mittelwert seiner Eltern ist.

Draco hörte mit offenem Mund zu. Wie in Merlins Namen hatten die Muggel all das herausgefunden? Könnten sie das Rezept \emph{sehen}?

„Jetzt,“ sagte Harry Potter, „nehmen wir an, dass wie bei Körpergröße es viele kleine Stellen im Rezept gibt, wo entweder „Magie“ oder „nicht Magie“ stehen kann. Wenn man genügend viele Stellen mit „Magie“ hat, ist man ein Zauberer, wenn man \emph{sehr} viele Stellen hat, ist man ein mächtiger Zauberer, mit zu wenigen ein Muggel und dazwischen ein Squib. Wenn zwei Squibs heiraten, sollten die meisten ihrer Kinder auch Squibs seinen, aber manchmal wird ein Kind Glück haben und die meisten Magie-Papiere des Vaters und die meisten Magie-Papiere der Mutter bekommen und damit stark genug sein um ein Zauberer zu sein. Aber vermutlich kein besonders mächtiger. Wenn du mit vielen starken Zauberer anfängst und sie nur einander heirateten, dann würden sie stark bleiben. Aber wenn sie anfingen Muggelstämmige zu heiraten, die nur gerade so magisch waren, oder Squibs… Verstehst du? Das Blut würde sich nicht perfekt vermischen, es wäre ein Glas Kieselsteine, nicht ein Glas Wasser, weil das die Art ist wie Blut funktioniert. Es würde immernoch dann und wann mächtige Zauberer geben, genau dann wenn sie mit Glück viele Magie-Papiere erhalten haben. Aber sie wären nicht mehr so stark, wie die stärksten Zauberer von früher.“

Draco nickte langsam. Er hatte es nie in dieser Art erklärt bekommen. Es hatte eine wunderliche Schönheit wie genau es passte.

„\emph{Aber,“} sagte Harry. „Das ist nur eine Hypothese. Nehmen wir an es gäbe nur \emph{eine} Stelle im Rezept , die jemanden zum Zauberer macht. Nur \emph{eine} Stelle an der ein Papierstück „Magie“ oder „nicht Magie“ sagen kann. Und es gibt immer zwei Versionen von allem. Dann gibt es nur drei Möglichkeiten. Beide Versionen könnten „Magie“ sagen. Eine Version könnte „Magie“ sagen und die andere Version „nicht Magie“. Oder beide Versionen sagen „nicht Magie“. Zauberer, Squibs und Muggel. Zwei Kopien „Magie“ und man kann Zauber wirken, eine Kopie und man kann noch Zaubertränke oder magische Geräte verwenden und keine Kopie bedeutet, dass man schon Probleme hat, Magie direkt anzuschauen. Muggelstämmige wären also nicht wirklich die Kinder von Muggeln, stattdessen werden sie von Squibs geboren, beide Elternteile mit einer „Magie“-Kopie, die in der Muggelwelt aufgewachsen sind. Stell dir jetzt vor eine Hexe heiratet einen Squib. Jedes Kind würde eine Stück Papier mit „Magie“ darauf von der Mutter bekommen, in jedem Fall, es macht keinen Unterschied welches Papier gewählt wird, beide sagen „Magie“. Aber wie wenn man eine Münze wirft, wird das Kind in der Hälfte der Fälle ein weiteres „Magie“- Papier vom Vater erhalten und sonst bekommt das Kind das „Nicht-Magie“-Papier des Vaters. Wenn also eine Hexe einen Squib heiratet, folgen als Resultat nicht sehr viele schwach magische Kinder. Die Hälfte der Kinder werden Zauberer und Hexen sein, vergleichbar stark wie ihre Mutter, und die andere Hälfte werden Squibs sein. Denn, wenn es nur eine Stelle im Rezept gibt, die jemanden zum Zauberer macht, dann ist Magie nicht wie ein Glas Kieselsteine, dass sich vermischen kann. Es ist wie ein einzelner magische Kieselsteine, der Stein des Zauberers.“

Harry ordete drei Papierpaare nebeneinander an. Auf ein Paar schrieb er „Magie“ und „Magie“. Bei einem anderen Paar schrieb er „Magie“ nur auf das obere Papier. Und das letzte Paar ließ er frei.

„In diesem Fall,“ sagte Harry, „hat man entweder zwei Steine oder eben nicht. Entweder man ist ein Zauberer oder nicht. Mächtige Zauberer würde zu diesen werden indem sie fließig lernen und viel üben. Falls Zauberer jedoch von Natur aus weniger mächtig werden, nicht weil Magie verloren geht , sondern weil sie die Zauber nicht einsetzen können… dann essen die vielleicht etwas Falsches oder sowas. Aber falls es über acht Jahrhunderte stetig schlechter geworden ist, dann könnte das bedeuten, dass die Magie selbst aus der Welt verblasst.“

Harry arrangierte ein weiteres Paar Papiere nebeneinander und nahm sich eine Feder. Nach kurzer Zeit stand auf je einem Papierstück „Magie“, während das andere frei blieb.

„Und das bringt mich zu meiner Vorhersage,“ sagte Harry. „Was passiert, wenn zwei Squibs heiraten. Wirf eine Münze zweimal. Es kann Kopf und Kopf, Kopf und Zahl, Zahl und Kopf und Zahl und Zahl vorkommen. Das heißst in einem Viertel der Fälle wirst du zweimal Kopf erhalten, in einem Viertel der Fälle wirst du zweimal Zahl erhalten und in der Hälfte der Fälle wirst du einmal Kopf und einmal Zahl erhalten. Ein Viertel der Kinder erhalten Magie und Magie und wären demnach Zauberer. Ein Viertel der Kinder erhalten nicht-Magie und nicht-Magie und wären daher Muggel. Die restliche Hälfte wären Squibs. Es ist ein sehr altes und sehr klassisches Muster. Es wurde von Gregor Mendel entdeckt, der nicht vergessen ist, und war der erste Hinweis überhaupt wie das Rezept funktioniert. Jeder der irgendwas über die Wissenschaft der Vererbung weiß, würde dieses Muster sofort erkennen. Es wäre nicht exakt, nicht mehr als wenn du vierzig-mal eine Münze zweifach wirfst du auch nicht immer genau zehnmal ein Paar von zweimal Kopf erhälst. Wenn aber zwischen sieben und dreizehn Zauberer unter den vierzig Kindern wären, dann ist das ein starker Indikator. Das ist der Test, den ich dir auftrug. Lass uns nun deine Daten anschauen.“

Und bevor Draco auch nur denken konnte, hatte Harry Potter das Pergament aus Dracos Hand genommen.

Dracos Kehle war sehr trocken.

Achtundzwanzig Kinder.

Er war sich über die genaue Zahl nicht sicher, doch er glaubte fest, dass ungefähr ein Viertel Zauberer gewesen waren.

„Sechs Zauberer aus achtundzwanzig Kindern,“ sagte Harry Potter nach einem Moment. „Hm, damit hat sich das. Und Erstklässler haben die gleichen Zauber mit der gleichen Stärke auch vor acht Jahrhunderten eingesetzt. Dein Test und mein Test kamen zum gleichen Ergebnis.

Es entstand eine lange Stille im Klassenzimmer.

„Was jetzt“ flüsterte Draco.

Er war noch nie so verängstigt gewesen.

„Es ist noch nicht definiv“, sagte Harry Potter. „Mein Experiment schlug fehl, weist du nicht mehr? Es ist notwendig, dass du noch einen weiteren Test gestaltest, Draco.“

„Ich, Ich….“, sagte Draco. Seine Stimme versagte. „Ich kann das nicht Harry, das ist zu viel für mich.“

Harrys Blick war unerschütterlich. „Doch du kannst, weil du musst. Ich habe selbst darüber nachgedacht, aber dann habe ich das über das Interdikt von Merlin herausgefunden. Draco, gibt es einen Weg die Stärke von Magie direkt zu beobachten? Ein Weg, der nichts mit dem Blut der Zauberer oder den gelernten Zaubern zu tun hat?“

Dracos Verstand war völlig leer.

„Etwas, dass die Magie betrifft, betrifft die Zauberer,“ sagte Harry. „Aber dann können wir nicht unterscheiden, ob es die Zauberer oder die Magie war. Was beeinflusst Magie, das \emph{kein Zauberer ist?“}

„Magische Geschöpfe, natürlich,“ sagte Draco ohne überhaupt darüber nachzudenken.

Harry Potter fing an zu lächeln. „Draco, das ist \emph{brilliant.“}

\emph{Es ist die Art von dummer Frage, die man nur in dem Fall stellen würde, wenn man bei Muggeln aufgewachsen wäre.}\\ Dann wurde das mulmige Gefühl in Dracos Magen schlimmer, als er bemerkte, was es bedeutete wenn magische Geschöpfe \emph{in der Tat schwächer würden . Dann würde sie endgültig wissen, dass die Magie verblasst und ein Teil von Draco war sich bereits sicher, dass sie genau das herausfinden würden. Er wollte es nicht sehen, er wollte es nicht \emph{wissen}…}

Harry Potter war schon halb aus der Tür. „Komm jetzt, Draco! Es gibt ein Portrait nicht weit von hier, wir fragen sie einfach nach jemand sehr altem und finden es sofort heraus! Wir tragen Umhänge, falls uns jemand sieht, rennen wir eben davon! Los jetzt!“

Danach dauerte es nicht mehr lang.

Es war ein breites Portrait, aber wegen der drei Personen darin, sah es durchaus überfüllt aus. Es gab einen Mann mittleren Altern aus dem zwölften Jahrhundert, gekleidet in schwarze Kleiderbahnen; der mit einer traurig aussehenden jungen Frau des vierzehnten Jahrhunderts sprach, deren Haar, scheinbar als ob es von einem Elektrostatikzauber aufgeladen worden wäre, sich über ihren Kopf kräuselte; die mit einem ehrwürdigen, runziglen, alten Man aus dem siebzehnten Jahrhundert sprach, der eine Krawatte aus purem Gold trug; und ihn konnten sie verstehen.

Sie hatten nach Dementoren gefragt.

Sie hatten nach Phönixen gefragt.

Sie hatten nach Drachen und Trollen und Hauselfen gefragt.

Harry runzelte die Stirn, und erklärte, dass nur Kreaturen, die die meiste Magie benötigten, komplett ausgestorben sein könnten, und dann nach den mächtigsten bekannten magischen Geschöpfen gefragt.

Es gab auf der Liste nichts Unerwartetes, außer einer Spezies dunkler Geschöpfe, namens Gedankenschinder (Mind Flayer, Dnd-Übersetzung ist mir unbekannt), bei der der Übersetzer anmerkte sie wären schließlich von Harold Shea ausgerottet worden, und die klangen nicht einmal halb so entsetzlich wie Dementoren.

Magische Geschöpfe waren wie es schien heute genauso machtig, wie sie es jemals waren.

Das mulmige Gefühl in Dracos Magen löste sich etwas, jetzt war er nur noch verwirrt.

„Harry,“ sagte Draco, in mitten einer Übersetzung der Liste aller elf Kräfte eines Betrachterauges, durch den alten Mann, „was bedeutet das?“

Harry hob einen Finger und der alte Mann beendete seine Liste.

Dann bedankte Harry sich bei allen Portraits für die Hilfe ---- Draco tat es, im Prinzip automatisch, ihm nach und dabei noch liebenswürdiger ---- und sie kehrten zum Klassenzimmer zurück.

Und Harry zog das ursprüngliche Pergament mit den Hypothesen hervor, und fing an zu schreiben.

Beobachtung:\\ \emph{Die Zaubererschaft ist nicht mehr so stark, wie zu der Zeit, in der Hogwarts gegründet wurde.}

Hypothesen:\\ \emph{1\emph{. Die Magie selbst verblasst.}}\\ \emph{2. Zauberer \emph{kreuzen sich mit Muggeln und Squibs.}}\\ \emph{3. Das Wissen um mächtige Zaubersprüche geht verloren.}\\ \emph{4. Zauberer essen während ihre Kindheit das falsche Essen, oder irgendwas anderes außer ihrem Blut lass sie schwächer heranwachsen.}\\ \emph{5. Die Muggeltechnologie beeinträchtigt die Magie. (Seit 800 Jahren?)}\\ \emph{6. Stärkere Zauberer haben weniger Kinder (Draco = Einzelkind? Finde heraus, ob drei mächtige Zauberer, Quirrell / Dumbledore / Dunkler Lord, Kinder hatten.)}

Tests:\\ \emph{A. Gibt es Zauber, die wir kennen aber nicht benutzen können (1 oder 2) oder sind die Zaubersprüche vergessen worden (3)?}\\ \emph{Resultat: Uneindeutig, wegen dem Interdikt von Merlin. Keine bekannten unnutzbaren Zauber, aber möglicherweise nur weil sie nicht weitergegeben wurden.}

\emph{B. Haben bereits vor langer Zeit die Erstklässler dieselbe Art Zauber mit der gleichen Stärke verwendet? (Schwaches Indiz für 1 im Vergleich zu 2, aber vielleicht verursacht Blut nur, dass mächtige Zauberei verloren geht.)}\\ \emph{Resultat: Dasselbe Level von Erstklässlerzaubern}

\emph{C. Zusätzlicher Test, der zwischen 1 und 2 mithilfe Wissenschaft der Vererbung unterscheidet, werde ich später erklären.}\\ \emph{Resultat: Es gibt nur einen Platz im Rezept, das jemanden zum Zauberer macht und entweder man hat zwei Stücke Papier mit „Magie“ darauf oder eben nicht.}

\emph{D. Verlieren magische Geschöpfe ihre Stärke? Unterscheidet 1 von (2 oder 3).}\\ \emph{Resultat: Magische Geschöpfe scheinen so stark wie sie jemals waren.}

\emph{„A schlug fehl,“ sagte Harry Potter. „B ist ein schwaches Indiz für 1 im Vergleich zu 2. C invalidert 2. D invalidert 1. 4 war unwahrscheinlich und B spricht außerdem gegen 4. 5 war unwahrscheinlich und D spricht dagegen. 6 wurde zusammen mit 2 widerlegt. Das bringt uns zu 3. Interdikt von Merlin oder nicht, ich habe wirklich keine bekannten Zauber gefunden, die man nicht einsetzen kann. Wenn man das alles einbezieht, sieht es so aus, als ob Wissen verloren geht.“}

Und die Falle schnappte zu.

\emph{Sobald die Panik verschwunden war, sobald Draco verstanden hatte, dass die Magie nicht verblasste, benötigte er ganze fünf Sekunden um zu bemerken.}

Draco richtete sich vom Tisch auf, er stand so schnell auf, dass sein Stuhl mit einem kratzenden\\ Geräusch über den Boden schlitterte und schließlich umfiel.

„Also war das alles nur ein blöder Trick.“

Harry starrte ihn, noch am Tisch sitzend, für einen Moment nur an. Als er sprach war seine Stimme leise. „Es war ein fairer Test, Draco. Falls ein anderes Ergebnis herausgekommen wäre, hätte ich es akzeptiert. Das hier ist nichts, wobei ich jemals betrügen würde. Jemals. Ich habe mir deine Daten nicht angeschaut, bevor ich meine Vorhersagen gemacht habe. Ich habe dir direkt erzählt, als das Interdikt von Merlin das erste Experiment invalidiert hat ----“

„Oh,“ sagte Draco, seine Stimme durchsetzt von Ärger „du wusstest nicht, wie dieses ganze Ding enden würde?“

\emph{„Ich wusste nichts, dass du nicht wusstest,“ sagte Harry, weiterhin leise. „Ich gebe zu, dass ich es vermutet habe. Hermine Granger war zu mächtig, sie hätte kaum magisch sein sollen, aber sie war es nicht, wie kann eine Muggelstämmige die Bestzaubernde von Hogwarts sein? Und auf ihre Aufsätze bekommt sie außerdem auch die besten Noten; das ist zu viel Zufall, dass dasselbe Mädchen die Stärkste im Bezug auf Magie und akademische Leistungen gleichzeitig ist, solange es nicht einen Grund gibt. Hermine Grangers Existenz deutete darauf hin, dass es nur eine Sache gibt, die jemanden zum Zauberer macht, etwas du entweder hast oder nicht hast, und der Stärkeunterschied nur darauf besteht, wie viel wir wissen und wie viel wir üben. Und es gab keine separaten Klassen für Reinblüter und Muggelstämmige und so weiter. Es gab zu viele Abweichungen dieser Welt zu der Welt, in der wir leben würde, wenn du Recht hättest. Aber Draco, ich habe nichts gesehen, dass du nicht auch gesehen hast. Ich habe keine Tests durchgeführt, von denen ich dir nicht erzählt habe. Ich habe nicht betrogen. Ich wollte mit dir zusammen die Antwort herausfinden. Und ich hätte noch nie daran gedacht, dass die Magie selbst aus der Welt verblassen könnte, bis du es vorgeschlagen hast. Auch für mich war das ein erschreckender Gedanke.“}

„Mir egal,“ sagte Draco. Er versuchte sehr seine Stimme zu kontrollieren, um nicht einfach anzufangen Harry anzuschreiben. „Du behauptest, du wirst nicht davonrennen und jemand anderem davon erzählen.“

„Nicht ohne es vorher mit dir abzusprechen,“ sagte Harry. Er öffnete seine Hände in einer flehenden Geste. „Draco, ich bin so nett wie ich sein kann, aber \emph{es hat sich herausgestellt, dass die Welt nicht in dieser Weise funktioniert.“}

„Schön. Dann sind du und ich jetzt durch. Ich werde einfach davongehen und vergessen was hier passiert ist.“

Draco wandte sich um, ein brennendes Gefühl in der Kehle, die Empfindung des Verrats und das war der Moment, indem er bemerkte, dass er Harry Potter wirklich gemocht \emph{hatte, doch dieser Gedanke verlangsamte ihn nicht, als er auf die Tür des Klassenzimmers zulief.}

Aus dem Klassenzimmer drang Harry Potters Stimme, diesmal lauter und besorgt:\\ „Draco… du \emph{kannst nicht vergessen. Verstehst du nicht? Das war dein Opfer.“}

Draco hielt mitten in der Bewegung an und drehte sich um. „\emph{Was redest du da?“}

Doch fühlte er bereits die klirrende Kälte in seinem Rückgrat.

Er wusste es bevor Harry Potter es sagte.

„Um ein Wissenschaftler zu werden. Du hast eine deiner Überzeugungen in Frage gestellt, nicht einmal eine unbedeutende Überzeugung, sondern etwas das für dich große Relevanz hat. Du hast Experimente gemacht, hast Daten gesammelt und das Resultat hat bewiesen, dass deine Überzeugung falsch war. Du hast das Ergebnis gesehen und verstanden, was es bedeutet .“ Harry Potters Stimme zitterte. „Denk daran, Draco, man kann eine \emph{richtige Überzeugung nicht in dieser Art opfern, da die Experimente statt sie zu widerlegen, sie bestätigen würden. Dein Opfer um Wissenschaftler zu werden war deine \emph{falsche} Überzeugung, dass sich Zauberer Blut mischt und schwächer wird.“}

„\emph{Das ist nicht wahr!“ sagte Draco. „Ich habe diese Überzeugung nicht geopfert, ich habe sie immer noch!“ Seine Stimme wurde lauter und die Kälte schlimmer.}

Harry Potter schüttelte seinen Kopf. Seine Stimme war nur ein Flüstern. „Draco… Es tut mir leid, Draco, du glaubst es \emph{nicht, nicht mehr.“ Harrys Stimme schwoll wieder an. „Ich werde es dir beweisen. Stell dir vor, jemand sagt dir er wurde einen Drachen in seinem Haus halten. Du forderst ihn zu sehen. Er sagt es ist ein unsichtbarer Drache. Du akzeptiert, und probierst hin zu hören. Er sagt es ist ein unhörbarer Drache. Du sagst, du würdest einfach ein bisschen Mehl in die Luft werfen und so die Umrisse des Drachen sehen. Er sagt der Drache ist im Bezug von Mehl durchlässig. Und das Detail, das ihn verrät, ist, dass er schon im Voraus weiß, genau welches experimentelle Resultat er erklären werden muss. Er weiß was herauskommen würde, wenn es dort keinen Drachen gibt, das heißt er weiß schon vorher, welche Ausrede er machen wird. Also sagt er da ist ein Drache. Vielleicht glaubt er sogar, dass er glaubt dass da ein Drache ist, es heißst Glaube- an-Glaube. Aber er glaubt es nicht wirklich. Man kann in dem was man glaubt einen Fehler machen , die meisten nehmen den Unterschied zwischen etwas glauben und denken, dass etwas zu glauben gut ist, nicht wahr.“ Harry Potter war vom Tisch aufgestanden und hatte einige Schritte auf Draco zu gemacht. „Und Draco, du glaubst nicht mehr an das Reinblütertum, ich werde dir zeigen, dass du es nicht mehr tust. Wenn das Reinblütertum richtig ist, dann ergibt Hermine Granger keinen Sinn, also wie erklärst du sie dir? Vielleicht ist sie eine Zaubererwaise, aufgezogen von Muggeln, so wie ich? Ich könnte zu Granger gehen und nach Bilder von ihren Eltern fragen, um zu überprüfen ob sie sich ähnlich sind. Würdest du erwarten, dass sie anders aussehen? Sollen wie diesen Test ausführen?“}

„Sie hätten sie zu Verwandten gebracht,“ sagte Draco, seine Stimme zitternd. „Sie würden sich dennoch ähnlich sein.“

„Siehst du. Du weißst schon, welches experimentelle Resultat du erklären werden musst. Wenn du noch immer an Reinblütertum glauben würdest, würdest du das akzeptieren und wetten das sie anders als ihre Eltern aussieht, da sie zu stark ist um eine echte Muggelstämmige zu sein ----“

„Sie \emph{hätten sie zu Verwandten gebracht!“}

„Wissenschaftler können Tests durchführen, um sicher zu gehen, dass jemand wirklich das Kind von einem Vater ist. Granger würde wahrscheinlich das machen, wenn ich ihrer Familie genug bezahle. \emph{Sie hätte keine Angst vor dem Ergebnis. Also, was erwartest du, was der Test sagen wird? Sag es mir, wenn wir ihn durchführen sollen. Doch du weißst bereits, was der Test sagen wird. Du wirst es immer wissen. Du wirst es niemals vergessen können. Vielleicht wünschst du dir an die Reinblüter-Hypothesen zu glauben, aber du wirst immer genau das erwarten, was passiert wenn es nur eine Stelle gibt, die jemanden zum Zauberer macht. Das war dein Opfer um Wissenschaftler zu werden.“}

Draco atmete unregelmäßig. „Verstehst du \emph{was du getan hast?“ Draco stürtzte nach vorne und packte Harry beim Kragen seines Umhangs. Seine Stimme schwoll zu einem Schreien an, es klang unaushaltbar laut in der Stille des abgeschlossenen Klassenzimmers. „\emph{Verstehst du was du getan hast?“}}\\ Harrys Stimme war unsicher. „Du hattest eine Überzeugung. Diese war falsch. Ich habe dir dabei geholfen das einzusehen. \emph{Was wahr ist, ist bereits so, es anzuerkennen, macht es nicht schlimmer ----“}

Die Finger an Dracos rechter Hand formten eine Faust und holten aus und fuhren unaufhaltsam nach oben und schlugen Harry Potters Kiefer so hart, sodass sein Körper zunächst in einen Tisch und dann zu Boden fiel.

„\emph{Idiot!“ schrie Draco. „\emph{Idiot! Idiot!“}}

\emph{„Draco,“ flüsterte Harry vom Boden, „Draco, es tut mir leid, ich habe nicht gedacht, es würde in den nächsten Monaten passieren, ich habe nicht erwartet, dass du so schnell als Wissenschaftler erwachen würdest, ich habe gedacht, ich würde mehr Zeit haben dich vorzubereiten, dir die Techniken beizubringen, die es weniger schmerzhaft machen, zuzugeben das man falsch lag ----“}

„Was ist mit Vater?“ sagte Draco. Seine Stimme zitterte in Rage. „Wolltest du auch \emph{ihn darauf vorbereiten oder kümmert es dich nicht, was danach passiert?“}

„Du darfst es ihm nicht sagen!“ rief Harry, seine Stimme erhob sich schockiert. „Er ist kein Wissenschaftler! Du hast es versprochen, Draco!“

Für einen Moment hatte der Gedanke, dass Vater es nicht wusste etwas Befreiendes.

Und dann wallte der richtige Ärger in ihm hoch.

„Also hast du geplant, dass ich ihn anlüge, ihm vormache, ich würde noch immer daran glauben,“ sagte Draco, die Stimme zitternd. „Ich werde ihn immer anlügen müssen und jetzt kann ich kein Todesser mehr werden und kann ihm nicht einmal sagen wieso.“

„Wenn dich dein Vater wirklich liebt,“ flüsterte Harry vom Boden, „wird er dich immer noch lieben auch wenn du kein Todesser wirst und für mich klingt es so als ob er dich wirklich liebt, Draco ----“

„\emph{Dein Stiefvater ist ein Wissenschaftler,“ sagte Draco. Die Worte drangen wie scharfe Messer aus seinem Mund. „Wenn \emph{du} kein Wissenschaftler wirst, würde er dich trotzdem lieben. Aber du wärst für ihn weniger besonders.“}

Harry zuckte zurück. Der Junge öffnete seinen Mund, vielleicht um etwas wie „Es tut mir leid“ zu sagen, schloss dann aber seinen Mund als ob er sich es besser überlegt hätte, was entweder sehr clever von ihm oder sehr glücklich für ihn war, denn dann hätte Draco ihn womöglich ermordet.

„Du hättest mich warnen sollen,“ sagte Draco. Seine Stimme schwoll an. „\emph{Du hättest mich warnen sollen!“}

„Ich… Ich habe dich gewarnt… jedes Mal wenn ich dir von dieser Macht erzählt habe, ich habe dir vom Preis erzählt. Ich sagte, du würdest zugeben müssen falsch zu liegen. Ich sagte das könnte das Schwierigste für dich sein. Das war das Opfer, welches jeder machen muss um ein Wissenschaftler zu werden. Ich fragte, was passieren würde, wenn die Experimente das eine und Familie und Freunde das andere sagen ----„

\emph{„\emph{Du nennst das eine Warnung?“} Draco schrie inswischen. \emph{„Du nennst das eine Warnung? Wenn wir ein Ritual machen, welches ein permanetes Opfer erfordert?“}}

\emph{„Ich… Ich…“ Der Junge auf dem Boden schluckte. „Ich fürchte ich habe mich nicht klar ausgedruckt. Es tut mir leid. Aber das was von der Wahrheit zerstört werden kann, sollte es auch.“}

Ihn zu schlagen war nicht mehr genug.

„In einer Sache liegst du falsch,“ sagte Draco, seine Stimme tödlich. „Granger ist nicht die stärkste Schlülerin in Hogwarts. Sie bekommt nur die besten Noten im Unterricht. Du wirst bald den Unterschied herausfinden.“

Plötzlicher Schock zeigte sich auf Harrys Gesicht und er versuchte sich schnell aufzurichten ----

Es war bereits zu spät für ihn.\\ „\emph{Expelliarmus!“}

Harrys Zauberstab flog durch quer durch den Raum.

„\emph{Gom jabbar!“}

Ein Strom von tintenähnlicher Dunkelheit umfasste Harrys linke Hand.

„Das ist ein Folterzauber,“ sagte Draco. „Er wird genutzt, um Informationen aus Menschen herauszubekommen. Ich werde ihn auf dir lassen und die Tür hinter mir versperren, wenn ich gehe. Vielleicht wähle ich den Abschließzauber sodass er nach einigen Stunden nachlässt. Vielleicht wird er auch nicht nachlassen bis zu hier drin verreckt bist. Viel Spaß.“

Draco bewegte sich flüssig rückwärts, den Zauberstab noch immer auf Harry. Dracos Hand griff nach unten, nahm seine Büchertasche hoch, ohne in seinem Zielen zu schwanken.

Der Schmerz zeigte sich schon auf Harry Potters Gesicht als er sprach. „Malfoys stehen über den Gesetzen zur Beschränkung der Zauberei Minderjähriger, habe ich Recht? Es ist nicht wegen stärkerem Blut. Es ist weil du bereits geübt hast. Am Anfang warst du so schwach wie jeder von uns. Ist meine Annahme falsch?“

Draco Hand schloss sich fester um seinen Zauberstab, sodass seine Knöcheln weiß hervortraten, doch er hielt seinen Zauberstab konstant auf ihn gerichtet.

„Nur das du es weißst,“ sagte Harry durch zusammengebissene Zähne, „wenn du mir erzählt hättest, ich läge falsch , hätte ich zugehört. \emph{Ich hätte dich nie gefoltert, wenn du mir gezeigt hättest, dass ich falsch liege. Und du wirst es auch. Eines Tages. Du bist jetzt als Wissenschaftler erwacht und selbst wenn du nie lernst deine Kräfte einzusetzen, wirst du immer,“ keuchte Harry, „nach Wegen suchen, deine Überzeugungen, zu überprüfen ----“}

Draco entfernte sich nun sehr viel weniger flüssig, dafür aber schneller, und musste sich anstrengenen, den Zauberstab weiterhin auf Harry zu richten, als er hinter sich griff um die Tür zu öffnen und aus dem Klassenzimmer zu treten.

Dann schloss Draco die Tür wieder.

Er wirkte den mächtigsten Verriegelungszauber, den er kannte.

Draco wartete bis er Harrys ersten Schrei hörte, bevor er \emph{Quietus wirkte\emph{.}}

Und dann ging er davon.

\emph{„Aaahhhhhhh! Finite Incantatem! Aaahhhhhhh!“}

Harrys linke Hand wurde in einen Topf siedenenes Öl gelegt und dort gelassen. Er hatte alles, was er noch in sich hatte in das \emph{Finite Incantatem gegeben, und dennoch keinen Erfolg gehabt.}

Einige Zauber benötigten spezifische Gegenzauber, sonst konnte man sie nicht außer Kraft setzen, oder vielleicht war Draco schlicht so viel stärker.\\ „Aaahhhhhhh!“

Harrys Hand begann ernsthaft zu schmerzen und das beeinträchtigte seine Versuche kreativ über das Problem nachzudenken.

Aber einige wenige Schreie später, realizierte Harry was er zu tun hatte.

Sein Beutel war unglücklicherweise auf der falschen Seite seines Körpers und er musste sich verrenken um in ihn hinein zu fassen, besonders mit seiner anderen Hand reflexartig fuchtelnd, in einem unaufhaltbaren Versuch die Schmerzensquelle abzuschütteln. Bis es ihm gelang, hatte sein anderer Arm es bereits geschafft seinen Zauberstab wieder weg zu werfen.

„Heil- aaahhhhh! Pack! Heilpack!“

Auf dem Boden, war das grüne Licht zu matt um etwas zu erkennen.

Harry konnte nicht stehen. Er konnte nicht kriechen. Er rollte über den Boden, an die Stelle an der er seinen Zauberstab vermutete, doch dort war er nicht, und es gelang ihm sich mit einer Hand aufzustützen, hoch genug um seinen Zauberstab zu finden und er rollte sich dorthin und nahm den Zauberstab und rollte zurück zum geöffneteten Heilpack. Die Aktion wurde von einigen Schreien und etwas Erbrechen begleitet.

Er benötigte acht Versuche bis er \emph{Lumos einsetzen konnte.}

Und dann, naja… das Packet war nicht dafür gedacht mit einer Hand geöffnet zu werden, weil alle Zauberer Idioten waren, darum. Harry musste seine Zähne verwenden, sodass es eine Weile dauerte bis Harry es schließlich schaffte, den Betäubungsstoff um seine linke Hand zu wickeln.

Nachdem jegliches Gefühl in seiner linken Hand verschwunden war, ließ Harry seine Gedanken auseinanderfallen, und lag bewegungslos auf dem Boden, und weinte eine Zeit lang.

\emph{Gut, sagte Harrys Verstand still in sich hinein, als er sich soweit erholt hatte, um wieder in Worten zu denken. \emph{War es das wert gewesen?}}

Langsam, streckte sich Harrys funktionierende Hand zum Tisch.

Harry zog sich auf die Füße .

Nahm einen tiefen Atemzug.

Atmete aus.

Lächelte.

Es war hatte nicht viel Ähnlichkeit mit einem wirklichen Lächeln, aber dennoch, ein Lächeln war es.

\emph{Danke, Professor Quirrell, ohne Sie hätte ich nicht verlieren können.}

Er hatte Draco nicht gedreht, er war nicht einmal nahe daran. Gegensätzlich zur Überzeugung Dracos, war er noch immer das Kind eines Todessers, durch und durch. Noch immer ein Junge, der in dem Glauben aufgewachsen war, „Vergewaltigung“ wäre was die coolen älteren Kinder machten. Aber dennoch, es war ein ausgezeichneter Start.

Harry konnte nicht vorgeben, es wäre alles nach Plan verlaufen. Es war alles komplett der Situation abhängig verlaufen. Der \emph{Plan hatte nichts dieser Art bis ungefähr Dezember vorgesehen, doch trotzdem hatte Harry Draco die Techniken beigebracht, Indizen nicht zu ignorieren, wenn er sie findet.}

Aber er hatte die Angst auf Dracos Gesicht gesehen, verstanden, dass Draco \emph{bereits alternative Hypothesen ernst nahm und den Moment ergriffen. Ein Fall echter Neugier hatte in der Rationalität eine ähnliche Macht der Überzeugung wie wahre Liebe in Filmen.}

Im Nachhinein, hatte Harry sich selbst Stunden gegeben um die wichtigste Entdeckung in der Geschichte der Magie zu machen und Monate um durch die unentwickelten Barrieren eines Elfjährigen zu dringen. Das könnte auf ein größeres kognitives Defizit Harrys hindeuten, nämlich die Zeitabschätzung bis zur Aufgabenerfüllung (Aufgabenerfüllungszeitabschätzung).

Würde er für seine Taten in die Wissenschaftlerhölle landen? Harry war sich nicht sicher. Er hatte arrangiert, dass Dracos Verstand sich mit der Möglichkeit des Verblassens der Magie beschäftigte, sicher gestellt, dass Draco diesen Teil des Experiments ausführte, der zunächst so wirkte als ob er in jene Richtung deutete. Er hatte bis nach seinen Erklärungen zu Genetik gewartet, um Draco dazu zu bringen über magischen Kreaturen nachzudenken (obwohl Harry eher an uralte Artifakte wie den Sprechenden Hut gedacht hatte, die nicht mehr dupliziert werden können, aber noch immer funktionieren). Aber Harry hatte wirklich keine Indizien übertrieben, nicht die Bedeutung irgendeines Tests verfälscht. Als das Interdikt von Merlin einen Test unbrauchbar gemacht hatte, der eigentlich entscheidend sein sollte, hatte er es Draco direkt mitgeteilt.

Und dann war da der Teil \emph{danach…}

Aber er hatte nicht wirklich Draco \emph{angelogen. Draco hatte es geglaubt und \emph{das würde es wahr machen.}}

Das Ende war zugegebenermaßen nicht lustig gewesen.

Harry drehte sich um und stolperte zur Tür.

Es war an der Zeit Dracos Verriegelungszauber zu testen.

Der erste Schritt war es einfach zu versuchen den Türknauf zu drehen. Draco hätte etwas vorgetäuscht haben können.

Draco hatte nichts vorgetäuscht.

„\emph{Finite Incantatem.“ Harrys Stimme klang eher heiser und er könnte das Scheitern des Zaubers fühlen.}

Also probierte er es erneut und diesmal fühlte es sich richtig an. Dennoch hatte er mit einem weiteren Drehen des Türknauf keinen Erfolg. Nicht überraschend.

Zeit die schweren Geschütze aufzufahren. Harry atmete tief ein. Dieser Zauber war einer der stärksten, die er bisher gelernt hatte.

„\emph{Alohomora!“}\\ Harry strauchelte nach dem Zauber ein wenig.

Und die Klassenzimmertür war noch immer nicht zu öffnen.

Das schockierte Harry. Harry hatte natürlich nicht vorgehabt sich auch nur in die Nähe von Dumbledores verbotenem Korridor zu begeben. Aber trotzdem hatte auf ihn ein Zauber zum Öffnen magischer Schlösser ziemlich nützlich gewirkt und so hatte Harry ihn gelernt. War Dumbledores verbotener Korridor dazu gedacht solche Leute anzuziehen die zu beschränkt waren um zu bemerken, dass die Sicherheitsmaßnahmen schwächer waren als die Maßnahmen die Draco Malfoy hätte ausführen können?

Die Furcht schlich sich wieder in Harrys System ein. Der Bepackzettel im Heilpack hatte gesagt, sicher könnte der Betäubungsstoff nur für 30 Minuten verwendet werden. Danach wurde es sich automatisch lösen und für 24 Stunden nicht wieder zu verwenden sein. Jetzt war es 18:51 Uhr. Er hatte den Betäubungstoff vor ungefähr fünf Minuten angelegt.

Also trat Harry einen Schritt zurück und betrachetete die Tür. Es war eine solide Planke aus dunklem Eichenholz, unterbrochen nur durch den Messingtürknauf.

Harry kannte keine explosiven oder schneidenen oder durchschlagenen Zauber und Explosives zu verwandeln, würde gegen die Regel der Verwandlung verstoßen, nichts zu verwandeln, dass verbrannt werden sollte. Säure wäre eine Flüssigkeit, würde also Dämpfe hervorrufen…

Aber das war kein Hindernis für einen \emph{kreativen Geist.}

Harry legte seinen Zauberstab gegen eines der Messingscharniere der Tür und konzentriete sich auf die Form von Baumwolle als ein reine Abtraktion abseits jeglichem Materials und auch auf das reine Material abseits des Musters des Messingscharniers und vereinigte die beiden Konzepte, gab Form und Substanz vor. Eine Stunde Übung in Verwandlung jeden Tag für einen Monat, erlaubten Harry inszwischen ein Subjekt von fünf Kubikzentimetern in etwas unter einer Minute zu verwandeln.

Nach zwei Minuten hatte sich der Scharnier nicht im Mindesten verändert.

Wer auch immer Dracos Verriegelungszauber erfunden hatte, schien auch daran gedacht zu haben. Oder die Tür war ein Teil von Hogwarts und das Schloss war schlicht immun.

Ein Blick genügte um die Wände als massiven Stein zu erkennen. Genauso der Boden. Genauso die Decke. Man konnte nicht einen Teil eines Ganzes separat verwandeln; Harry müsste versuchen die gesamte Wand zu verwandeln, was einige Stunden wenn nicht Tage kontinuierlicher Arbeit bedeuten würde, wenn er es überhaupt vollbringen könnte. Und wenn die Wand nicht mit dem Rest des Schlosses zusammenhängt…

Harrys Zeitumkehrer würde sich nicht bis nach 21 Uhr öffnen. Danach könnte er zurück zu 18 Uhr reisen, bevor die Tür verriegelt wurde.

Wie lang wurde der Folterzauber halten?

Harry schluckte schwer. Tränen traten ihm wieder in die Augen.

Sein brillianter, kreativer Verstand hatte soeben den genialen Vorschlag vorgebracht, Harry könne seine Hand mithilfe der Metallsäge aus der Werkzeugkiste, die in seinem Beutel verwahrt war, abschneiden, was weh tun würde, aber vielleicht weniger als Draco Schmerzenszauber, da seine Nerven verschwunden wären; und außerdem hatte er Verbände in seinem Heilpack.

Und das offentsichtlich eine schreckliche dumme Idee, die Harry sein ganzes Leben lang bereuen würde.

Aber Harry wusste nicht, ob es zwei Stunden unter Folter aushalten würde.

Er wollte aus diesem Klassenzimmer \emph{heraus, er wollte \emph{jetzt} aus dem Klassenzimmer heraus, er wollte nicht hier drin zwei Stunden schreiend auf den Moment warten, indem er seinen Zeitumkehrer benutzen konnte, er musste \emph{hier raus} und jemanden finden, der seine Hand vom Folterzauber befreien konnte…}

\emph{Denk nach! Schrie Harry sein Gehirn an. Denk nach! Denk nach!}

Der Schlafsaal der Slytherins war fast leer. Sie waren beim Abendessen. Aus irgendeinem Grund fühlte sich Draco nicht gerade hungrig.

Draco zog die Tür zu seinen privaten Schlafraum zu, schloss ab, verriegelte sie mit einem Zauber, '\emph{Quietus'te sie, setzte sich auf sein Bett und fing an zu weinen.}

Es war nicht fair.

Es war nicht fair.

Es war das erste Mal, dass Draco jemals wirklich \emph{verloren hatte. Zuvor hatte sein Vater ihn davor gewarnt, wie schmerzhaft es sein würde, das erste Mal wirklich zu verlieren, aber er hatte \emph{so viel} verloren, es war nicht fair, es war nicht fair für ihm, \emph{alles} gleich bei der ersten Niederlage zu verlieren.}

Irgendwo in den Kerkern, schrie ein Junge, den Draco tatsächlich gemocht hatte, vor Schmerzen. Draco hatte nie zuvor jemandem Schmerzen zugefügt, den er gemocht hatte. Leute zu bestrafen, die es verdient hatten, sollte Spaß machen, aber dies fühlte sich einfach nur krank an. Vater hatte ihn davor gewarnt und Draco fragte sich ob dies eine harte Lektion war, die jeder durchmachte während er aufwuchs, oder ob er schlicht schwach war.

Draco wünschte sich es wäre Pansy, die schrie. Das würde sich besser anfühlen.

Und der schlimmste Teil war das Wissen, dass es ein Fehler gewesen sein könnte Harry Potter zu verletzen.

Wer sonst war jetzt noch für ihn übrig? Dumbledore? Nach dem was er getan hatte? Draco wäre eher lebendig verbrannt worden.

Draco würde zu Harry Potter zurückkehren müssen, weil es keinen anderen Platz mehr für ihn gab. Und wenn Harry Potter sagte, er wolle ihn nicht, dann wäre Draco ein Niemand, nur ein erbärmlicher kleiner Junge, der niemals ein Todesser sein konnte, der niemals Dumbledores Fraktion beitreten konnte, der niemals Wissenschaft erlernen konnte.

Die Falle war perfekt vorbereitet und perfekt ausgeführt worden. Vater hatte Draco immer wieder gewarnt; wenn etwas in dunklen Ritualen geopfert wurde, konnte es nicht mehr zurückgewonnen werden. Aber sein Vater hatte nicht gewusst, dass die verfluchten Muggel Rituale entwickelt hatten, die keine Zauberstäbe benötigten, die dich ohne dein eigenes Wissen zur Durchführung verleiten konnten und was nur eines der schrecklichen Geheimnisse der Wissenschaftler war, welches Harry Potter an ihn weitergeben hatte.

Draco weinte nun heftiger.

Er wollte dies nicht, er \emph{wollte das nicht aber es gab kein zurück. Es war zu spät. Er war bereits ein Wissenschaftler.}

Draco wusste, er sollte zurück gehen, Harry Potter befreien und sich entschuldigen. Das wäre die weise Handlung.

Stattdessen blieb Draco in seinem Bett und schluchzte.

Er hatte Harry Potter schon Schmerzen zugefügt. Es war möglicherweise die einzige Gelegenheit für Draco ihm jemals Schmerzen zuzufügen und er würde sich an diese Erinnerung für den Rest seines Lebens klammern.

Lass ihn weiter schreien.

Harry ließ die Reste seiner Metallsäge zu Boden fallen. Die Messingscharniere haben sich als undurchdringbar herausgestellt, sodass es nicht mal einen Kratzer gab und in Harry kam die Vermutung auf, selbst die Verzweiflungstat Säure oder Explosives zu verwandeln, hätte keinen Erfolg beim Öffnen der Tür. Auf der positiven Seite, hatte sein Versuch die Metallsäge zerstört.

Seine Uhr zeigte 19:02, mit weniger als fünfzehn Minuten übrig, und Harry versuchte sich zu erinnern, ob es noch andere scharfe Gegenstände in seinem Beutel gab, die er zerstören sollte, und fühlte dabei, wie weitere Tränen in ihm aufwallten. Wenn er doch nur, nachdem sein Zeitumkehrer sich öffnete, zurückreisen könnte und dies \emph{verhinder----}

In diesem Moment bemerkte Harry, wie \emph{blöd er sich verhielt.}

Dies war nicht das erste Mal, indem er in einem Raum eingeschlossen war.

Professor McGonagall hatte ihm bereits den korrekten Weg erklärt.

… sie hatte ihn angewiesen den Zeitumkehrer nicht für solche Sachen zu benutzen.

Würde Professor McGonagall verstehen, dass bei dieser Gelegenheit ein spezielle Ausnahme rechtfertigt ist? Oder würde sie einfach den Zeitumkehrer komplett wegnehmen?

Harry sammelte alle seine Sachen, alle Indizen, zusammen, und legte sie in seinen Beutel. Ein \emph{Scourgify kümmerte sich um das Erbrochene auf dem Boden, jedoch nicht um den Schweiß, der seine Umhänge durchtränkte. Er ließ die Tische umgeschoßenen, es war nicht wichtig genug es mit nur einer Hand zu machen.}

Als er fertig war, schaute Harry auf seine Uhr. 19:04 Uhr.

Und dann wartete Harry. Sekunden vergangen, doch es fühlte sich wie Jahre an.

Um 19:07 Uhr, öffnete sich die Tür.

Professor Flitwicks Rauschebart-Gesicht sah eher besorgt aus. „Geht es dir gut, Harry?“, sagte die quitschende Stimme des Hauslehrers von Ravenclaw. „Ich habe eine Nachricht erhalten, du wärst hier eingeschlossen----“

