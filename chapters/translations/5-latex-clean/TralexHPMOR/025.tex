

\hypertarget{warten-bevor-man-luxf6sungen-vorschluxe4gt}{% \section{4. Warten bevor man Lösungen vorschlägt}\label{warten-bevor-man-luxf6sungen-vorschluxe4gt}}

Kapitel 4

Warten bevor man Lösungen vorschlägt

Act: 2

(Die Sonne scheinte strahlend hell von der verzauberten Himmels-D ecke in die große Halle, erleuchtete die Schüler wie wenn sie unter freiem Himmel säßen, glänzte von den Tellern und Schüsseln wieder, als sie, vom Schlaf erholt, ihr Frühstück einnahm en, als Vorbereitung für welche Pläne auch immer sie für den Sontag gemacht hatten.)

So. Es gab nur eine Sache, die jemanden zum Zauberer macht.

Das war nicht überraschend, wenn man darüber nachdachte. Was DNA hauptsächlich machte, war Ribosomen zu erzählen wie sie Aminosäuren zusammen verketten sollten, sodass daraus Proteine entstanden. Konventionelle Physik schrieb vor, dass man hier niemals, niemals Magie heraus bekommen sollte.

Und dennoch schien Magie erblich zu sein, der DNA folgend.

Dann passierte dies wahrscheinlich \emph{nicht,} weil die DNA nichtmagische Aminosäuren zu magischen Proteinen verknüpfte.

Eher gab die Schlüssel-DNA-Sequenz überhauptkeine Magie von sich aus.

Die Magie kam von woanders.

(Am Ravenclaw-Tisch gab es einen Junge, der in den Raum starrte, während seine rechte Hand automatisch irgendeine unbedeutende Art Essen in seinen Mund löffelte, von was auch immer vor ihm war. Man hatte es wahrscheinlich mit einen Haufen Erde ersetzen können und er hätte es nicht bemerkt.)

Und aus irgendeinem Grund achtete die Quelle der Magie auf einen bestimmten DNA-Marker unter den Individuen, die sonst vollig normale vom Affen abstammende Menschen waren.

(In Wirklichkeit gab es ziemlich viele Jungen und Mädchen, die in den Raum starrten. Es war schließlich der Ravenclaw-Tisch.)

Es gab andere Begründungswege, die zur selben Schlussfolgerung führten. \emph{Komplexe} Maschineriewar stets universell innerhalb einer sexuell fortpflanzenden Spezies. Falls Gen B auf Gen A angewiesen war, dann musste A schon alleine nützlich sein und im Genpool selbstständig (annähernd) universell werden, bevor häufig genug B nützlich wäre, um einen Fitnessvorteil zu deutlich werden zu lassen. Dann, sobald B universell war, würde man eine Variante A* erhalten, die auf B angewiesen war, und dann C, das auf A* und B angewiesen war, dann B*, das auf C angewiesen war, bis schließlich die gesamte Maschiene auseinander fallen würde, wenn man auch nur ein einziges Stück entfernte. Doch es musste alles \emph{schrittweise} passieren -- die Evolution denkt nicht vorausschauend, sie würde nie B als Vorbereitung auf A späteres Universell-Werden vorantreiben.

Evolution war der simple historische Fakt, dass welche Organismen auch immer wirklich die meiste Nachkommen hatten, deren Gene demnach häufiger in der nächsten Generation auftreten würden. So musste jedes Teil der komplexen Maschinerie zunächst fast universell werden, bevor in der Maschiene sich andere Teile entwickeln konnten, die von der Präsenz des anderen abhängig waren.

Also war \emph{komplexe, voneinander abhängige} Maschinerie, die mächtigen, ausgeklügelten Maschienen, die das Leben steuerten, stets \emph{universell} innerhalb einer sexuell fortpflanzenden Spezies -- außer einer Handvoll \emph{nicht} voneinander abhängiger \emph{Varianten}, von denen zu jedem Zeitpunkt eine ausgewählt wurde, während weiterführende Komplexität festgelegt wurde. Deshalb hatten alle Menschen den gleichen grundlegenden Gehirnaufbau, die gleichen Emotionen, die gleiche Gesichtsausdrücke mit diesen Emotionen verknüpft; diese Anpassungen waren komplex, demnach \emph{mussten} sie universell sein.

Falls Magie genauso wäre, eine große, komplexe Anpassung mit vielen notwendigen Genen, dann hätten ein Paar aus einem Zauberer und einem Muggel ein Kind mit nur der Hälfte dieser Teile und die halbe Maschine würde nicht viel tun.

Und es hätte niemals Muggelstämmige gegeben. Selbst wenn alle Teile unabhängig von einander in den Muggel-Genpool gelangt wären, würden sie sich nie wieder in einem Platz zusammensetzen, um einen Zauberer zu bilden.

Es gab kein genetisch isoliertes Tal mit Menschen, die auf einen evolutionären Pfad gestolpert waren, der zu ausgeklügelten magischen Sektionen des Gehirn führte. Diese komplexe genetische Maschienerie, hätte, falls Zauberer sich mit Muggeln kreuzten, sich niemals wieder in einem Muggelstämmigen zusammengesetzt.

Wie auch immer Gene jemanden zum Zauberer machten, es \emph{passierte nicht} durcheinen enthaltenen Entwurf für komplexe Maschienerie.

Das war der andere Grund warum Harry das Mendel'sche Muster vermutet hatte. Wenn magische Gene nicht kompliziert waren, warum sollten es dann mehr als eines sein?

Und dennoch wirkte die Magie selbst ziemlich kompliziert. Ein Verriegelungszauber würde das Öffnen der Tür verhindern \emph{und} das Verwandeln der Scharniere verhindern \emph{und} \emph{Finite Incantatem} und \emph{Alohomora} widerstehen. Viele Elemente, alle in gleiche Richtung deutend: man könnte es Zielorientierung nennen, oder in einfacher Sprache, Zweckmäßigkeit.

Es gab nur zwei bekannte Ursachen für zweckmäßige Komplexität. Natürliche Selektion, die Dinge wie Schmetterlinge produzierte. Und intelligentes Ingenieurswesen, das Dinge wie Autos produzierte.

Magie schien nicht wie etwas, dass selbstrepliziertend entstanden wäre. Zauber waren zweckmäßig kompliziert, aber nicht, wie ein Schmetterling, kompliziert für den Zweck des sich selbst Kopierens. Zauber waren kompliziert für den Zweck dem Benutzer zu dienen, wie bei einem Auto.

Ein intelligenter Ingenieur hatte dann die Quelle der Magie erschaffen und ihr aufgetragen auf einen bestimmten DNA-Marker zu achten.

Der nächste naheliegende Gedanke war, dass dies mit etwas mit „Atlantis“ zu tun hatte.

Harry hatte Hermine zuvor danach gefragt \later im Zug nach Hogwarts, nachdem er gehört hatte, wie Draco es erwähnte hatte \later und soweit sie wusste, war nicht mehr als der Name selbst bekannt.

Es könnte eine reine Legende gewesen sein. Aber es war auch plausibel genug, dass eine Zivilisation von Magienutzern, besonders jene \emph{vor} dem Interdikt von Merlin, es geschafft hätte sich selbst hochzujagen.

Dieser Gedankengang konnte weitergeführt werden: Atlantis war eine isolierte Zivilisation, die irgendwie die Quelle der Magie erschaffen hatte und ihr aufgetragen hatte nur Menschen mit dem Atlantischen genetischen Marker zu dienen, dem Blut von Atlantis.

Und mit der gleichen Logik folgte, dass die gesprochenen Worte eines Zauberers, die Bewegungen des Zauberstabs, in sich selbst nicht komplex genug waren, um die Effekte eines Zaubers von Grund auf zu errichten \later nicht in der Art wie drei Milliarden Basenpaare menschlicher DNA kompliziert genug waren, um einen menschlichen Körper von Grund auf zu errichten, nicht in der Art wie Computerprogramme Tausende Bytes an Daten einnahmen.

Also waren die Worte und Zauberstabbewegungen nur Auslöser, gezogene Hebel an einer versteckten und komplexeren Maschiene. Knöpfe, keine Vorlagen.

Und ganz genau wie bei einem Computerprogramm, das nicht kompliliert wenn man auch nur einen einzigen Rechtschreibfehler gemacht hatte, würde die Quelle der Magie nicht reagieren, solange man den Zauber nicht in genau der richtigen Art und Weise gewirkt hatte.

Die Argumentationskette war unaufhaltsam.

Und sie führte unvermeidlich zu einer einzigen endgültigen Schlussfolgerung.

Die antiken Vorfahren der Zauberer, tausende von Jahren zuvor, hatten der Quelle der Magie aufgetragen nur dann Dinge schweben zu lassen, wenn jemand…

'Wingardium Leviosa' sagte.

Harry sackte am Frühstückstisch zusammen, seine Stirn erschöpft auf seine rechten Hand ruhend.

Es gab da eine Geschichte aus den Geburtsstunden der künstlichen Intelligenz \later damals, als sie gerade angefangen hatten und noch niemand die Ausmaße des Problems erkannt hatte \later über einen Professor, der einen seiner Diplom-Studenten dafür abgestellt hatte, das Problem des maschienellem Sehens zu lösen.

Harry fing an zu begreifen, wie sich dieser Diplom-Student gefühlt haben musste.

Das könnte eine Weile dauern.

Wieso war ein Aufwand nötig, um den Alohomora-Zauber auszuführen, wenn es doch wie das Drücken eines Knopfes war?

Wer war dumm genug gewesen, einen Zauber wie \emph{Avada Kedavra} einzubauen, der nur mithilfe von Hass eingesetzt werden konnte?

Warum benötigte wortlose Verwandlung eine vollständige mentaler Trennung vom Konzept der Form und vom Konzept des Materials?

Harry würde vielleicht noch nicht mit diesem Problem fertig sein, wenn er Hogwarts abschloss.

Er könnte noch immer an diesem Problem arbeiten, wenn er \emph{dreißigJahre alt} war. Hermine hatte Recht gehabt, Harry \emph{hatte} dies noch nicht auf tieferer Ebene gemerkt gehabt. Er hatte nur eine inspirierende Rede über Entschlossenheit gehalten.

Harrys Verstand erwägte kurz, ob ich darauf einstellen sollte, dass er das Problem vielleicht niemals lösen würde, entschied sich jedoch dann dafür, das würde zu weit gehen.

Außerdem, solange er innerhalb der ersten paar Jahrhunderten bis zur Unsterblichkeit gelangte, wäre alles gut.

Welche Methode hatte der Dunkle Lord verwendet? Jetzt wo er daran dachte, der Fakt, dass der Dunkle Lord es irgendwie geschafft hatte den Tod seines ersten Körpers zu überleben, war \emph{unendlich} viel wichtiger als der Fakt, dass er versucht hatte das magischen Britannien zu übernehmen \later

„Entschuldigen Sie,“ sagte eine erwartete Stimme hinter ihm in einem sehr unerwarteten Ton.

„Wenn es Ihnen passt, Mr~Malfoy wünscht eine Konversation.“

Harry verschuckte sich nicht an seinem Müsli. „Meinst du nicht 'Da Boss will mit dir reden'?“

Mr~Crabbe sah nicht glücklich aus. „Mr~Malfoy hat mir aufgetragen ordentlich zu sprechen.“

„Ich kann dich nicht hören,“ sagte Harry. „Du sprichst nicht ordentlich.“ Er wandte sich wieder seiner Schale kleiner blauer Kristallschneeflocken zu und aß bewusst noch einen Bissen.

'Da Boss will mit dir reden', erklang eine drohende Stimme hinter ihm. „Du kommt besser mit zu ihm, wenn du weißs was gut für dich is.“

Da. \emph{Jetzt} ging alles nach Plan.

Act 1:

„Ein \emph{Grund?“} fragte der alte Zauberer. Er hielt den Zorn von seinem Gesicht fern. Der Junge vor ihm war das Opfer gewesen und musste sicherlich nicht noch mehr eingeschüchtert werden. „Es gibt hierfür \emph{keine} Entschuldigung \later“

„Was ich ihm angetan hatte war schlimmer.“

Der alte Zauberer versteifte sich in plötzlichem Entsetzen. „Harry, \emph{was hast du getan?“}

„Ich habe Draco getäuscht, sodass er glaubt ich hätte derart ihn getäuscht, dass er an einem Ritual teilgenommen hatte, welches seinen Glauben an die Überzeugungen der Reinblüter geopfert hatte. Und das bedeutete, dass er niemals ein Todesser werden konnte, wenn er erwachsen wird. Er hat alles verloren, Schulleiter.“

Es entstand eine lange Stille im Büro des Schulleiters, unterbrochen nur von den winzigen Knall- und Pfeilgeräusche der kniffligen Geräte, die aber nach genügend vergangener Zeit wie Stille erschienen.

„Ach du liebe Zeit!“ rief der alte Zauberer, „Ich fühle mich nun \emph{wirklich} doof. Ich saß hier herum, annehmend du würdest versuchen den Erben der Malfoy zu rehabilitieren indem du ihm, sagen wir, \emph{Freundschaft und Güte zeigst.“}

„\emph{Ha!} Ja, als ob \emph{das} funktioniert hätte.“

Der alte Zauberer seufzte. Das trieb es etwas zu weit. „Sag mir, Harry; Kam es dir überhaupt \emph{in den Sinn,} dass es unpassend wäre zu versuchen jemanden mithilfe von Lügen und Täuschungen zu erretten?“

„Ich habe es geschafft ohne direkte Lügen zu erzählen und da wir hier über Draco Malfoy sprechen, denke ich das Wort nachdem Sie suchen ist \emph{passend.“} Der Junge wirkte ziemlich selbstzufrieden.

Der alte Zauberer schüttelte verzweifelt seinen Kopf. „Und \emph{das} ist der Held. Wir sind alle verloren.“

Act 5:

Der lange, enge Tunnel aus unbehauenem Gestein, nur erleuchtet vom Zauberstab eines Kindes, schien sich für Meilen zu erstrecken.

Der Grund hierfür war einfach. Er erstreckte sich für Meilen.

Es war 3 Uhr morgens und Fred und George liefen den langen Weg durch den Geheimgang, der von der Statue eines einäugigen Zauberers in Hogwarts bis zum Keller vom Honigtopf-Süßwarenladen in Hogsmeade führte.

„Wie läuft es den so?“ fragte Fred mit tiefen Stimme.

(Nicht das jemand zuhörte, dennoch hatte es etwas merkwürdiges mit normalen Stimme zu reden, während man durch einen Geheimgang ging.)

„Noch immer kaputt,“ erwiderte George.

„Beide, oder \later“

„Zwischenzeitlich hat eines sich selbst repariert. Das andere ist so wie immer.“

Die Karte war ein außerordentlich mächtigtes Artefakt, befähigt jedes einzelnes fühlende Wesen innerhalb der Ländereien von Hogwarts aufzuzeichnen, in Echtzeit, mit Namen. Fast garantiert wurde sie während des ursprünglichen Errichtens von Hogwarts erstellt. Es war \emph{nicht gut,} dass Fehler anfingen aufzutauchen. Die Chancen standen gut, dass niemand außer Dumbledore sie reparieren konnte, sollte sie defekt sein.

Und die Weasley-Zwillinge würden nie die Karte an Dumbledore weitergeben. Es wäre eine unverzeihliche Beleidigung an die Rumtreiber \later die vier Unbekannten, die es bewerkstelligt hatten einen Teil von \emph{Hogwarts Sicherheitssystem} zu stehlen, wahrscheinlich etwas von Slytherin selbst selbst Erstelltes und es in ein \emph{Werkzeug für Schülerstreiche} verwandelt hatten.

Manche würden dies vielleicht für respektlos halten.

Manche würden dies vielleicht für kriminell halten.

Die Weasley-Zwillinge glaubten jedoch fest daran, dass Godric Gryffindor es genehmigt hätte, wenn er es sehen könnte.

Die Brüder liefen weiter und immer weiter, meistens schweigend. Die Weasley-Zwillinge sprachen miteinander, wenn sie sich neue Streiche ausdachten oder wenn einer etwas wusste, das dem anderen unbekannt war. Ansonsten gab es kaum einen Grund. Wenn sie bereits die gleichen Informationen kannten, neigten sie dazu die gleichen Gedanken zu haben und die gleichen Entscheidungen zu treffen.

(Damals, vor langer Zeit, war es üblich gewesen, immer wenn magische eineiige Zwillinge geboren wurden, einen der beiden nach der Geburt zu töten.)

Nach einiger Zeit stiegen Fred und George aus dem Geheimgang in den staubigen Keller, übersät mit Fässern und Regalen gefüllt mit seltsamen Zutaten.

Fred und George warteten. Es wäre unhöflich gewesen etwas anderes zu tun.

Einen Moment später kletterte ein dünner, alter Man gekleidet in einen schwarzen Schlafanzug die Treppenstufen herunter und gähnte. „Hallo, Jungs“, sagte Ambrosius Flume. „Ich habe euch heute nacht nicht erwartet. Schon ausverkauft?“

Fred und George entschieden, dass Fred sprechen würde.

„Nicht ganz, Mr~Flume,“ sagte Fred. „Wir hatten gehofft, Sie könnten uns bei etwas weitaus… Interessanterem helfen.

„Nun, Jungs,“ sagte Flume, ernst klingend, „Ich hoffe ihr habt mich nicht geweckt, nur damit euch erneut versichern kann, dass ich euch keine Ware verkaufe, die euch in Gefahr bringen könnte. Jedenfalls nicht bevor ihr sechszehn seit \later“

George zog einen Gegenstand aus seinem Umhang hervor und reichte ihn wortlos an Flume weiter. „Haben Sie das gesehen?“ sagte Fred.

Flume betrachtete die gestrige Ausgabe des \emph{Tagespropheten} und nickte mürrisch. Die Überschrift auf der Zeitung lautete DER NÄCHSTE DUNKLE LORD? Und zeigte einen kleinen Jungen, den die Kamera eines Schüler mit einen ungewöhnlich grimigen und kaltem Gesichtsausdruck festgehalten hatte.

„Ich kann diesen Malfoy nicht fassen,“ schimpfte Flume. „Dem Jungen nachstellen, obwohl er gerade mal elf ist! Dieser Man sollte erfasst werden und verwendet werden um Schokolade herzustellen!“

Fred und George blinzelte zeitgleich. \emph{Malfoy} steckte hinter Rita Kimmkorn? Harry Potter hatte sie davor nicht gewarnt… was sicherlich bedeutete, dass Harry es nicht wusste. Er hätte sie niemals hineingezogen, wenn doch…

Fred und George tauschten Blicke. Naja, Harry \emph{musste} es nicht wissen, bis nachdem der Job erledigt war.

„Mr~Flume,“ sagte Fred leise, „der Junge-der-überlebte benötigt ihre Hilfe.“

Mr~Flume starrte sie beide an.

Dann ließ er seinen Atem mit einem Seufzer entweichen.

„In Ordnung,“ rief Flume, „was wollte ihr?“

Act 6:

Wenn Rita Kimmkorn auf eine saftige Beute konzentriert war, neigte sie dazu die wuselden Ameisen zu übersehen, die den Rest des Universum ausmachten, was wohl der Grund war warum sie mit dem kahl werdenen jungen Mann zusammenstieß, der in ihren Weg getreten war.

„Miss~Kimmkorn,“ sagte der Mann, ziemlich ernst und kalt klingend, für jemanden, der so ein junges Gesicht hatte. „Schön, Sie hier zu treffen.“

„Aus dem Weg, Bursche!“ blaffte Rita und versuchte an ihm vorbei zu schreiten. Der Mann in ihrem Weg replizierte ihre Bewegung derart perfekt, dass es so schien als ob keiner der beiden sich überhaupt bewegt hatte, als ob sie nur stillgestanden hatten, während sich die Straße unter ihnen verschob.

Ritas Augen verengten sich. „Für wen hältst du dich eigentlich?“

„Wie unfassbar töricht,“ sagte der Mann trocken. „Es wäre weise sich das Gesicht des getarnten Todessers zu merken, der Harry Potter ausbildet damit er der nächste Dunkle Lord wird. Letztendlich“ ein dünnes Lächeln, „klingt \emph{das} nicht wie jemand, den man gerne auf der Straße treffen würde, besonders nachdem man einen üblen Verriss über ihn für die Zeitung gedruckt hat.“

Rita brauchte einen Moment um die Referenz einzuordnen. \emph{Das} war Quirinius Quirrell? Er sah zu jung und zu alt zur gleichen Zeit aus; sein Gesicht, wenn es einmal den ernsten und herablassenden Ausdruck ablegte, würde zu jemandem in seinem späten Dreißigern gehören. Und sein Haar fiel bereits aus? Konnte er sich keinen Heiler leisten?

Nein, das was unwichtig, es gab eine Zeit und einen Platz an dem man ein Käfer sein wollte. Sie hatte gerade einen anonymen Hinweis erhalten, dass Madam Bones es mit einem ihrer jüngeren Assistenten trieb. Das würde einen außerordentlichen Bonus wert sein, falls es ihr gelang dies zu bestätigen, Bones stand weit oben auf der Abschussliste.

Der Informant hatte gesagt, dass Bones und ihr junger Assistent zum Mittagessen in einem speziellen Raum bei Mary's Place verabredet waren, ein sehr beliebter Raum für bestimmte Zwecke; ein Raum, der, wie sie herausgefunden hatte, gegen sämtliche Abhörgeräte abgesichert war, aber nicht gegen einen hübschen blauen Käfer gesichert war, der sich an eine Wand schmiegte…

„Aus dem \emph{Weg}!“ rief Rita, und versuchte Quirrell aus ihrem Pfad zu schieben. Quirells Arm schreifte abwehrend ihren eigenen, und Rita stolperte als ihr Schwung danaben ging.

Quirrell zog den linken Ärmel seines Umhangs hoch, seinen linken Arm offenbarent. „Beachten Sie,“ sagte Quirrell, „kein Dunkles Mal. Ich möchte, dass ihre Zeitung einen Widerruf veröffentlicht.“

Rita ließ einen ungläubiges Lachen heraus. Natürlich war der Mann kein richtiger Todesser. Die Zeitung hätte es nicht veröffentlicht wenn es so wäre. „Vergiss es, Bursche. Jetzt, verschwinde.“

Quirrell starrte sie für einen Moment an.

Dann lächelte er.

„Miss~Kimmkorn,“ sagte Quirrell, „Ich hätte gehofft einen Hebel zu finden, der sich als überzeugend erweist. Dennoch zeigt sich, dass ich mir nicht das Vergnügen verwehren kann, Sie einfach zu zerquetschen.“

„Viele haben es versucht. Jetzt geh aus meinem Weg, Bursche, oder ich suche ein paar Auroren und lasse dich für Behinderung des Journalismus einsperren.“

Quirrell gab ihr eine kleine Verbeugung und schritt dann weiter. „Auf Wiedersehen, Rita Kimmkorn,“ sagte die Stimme hinter ihr.

Als Rita vorwärts drängte, bemerkte sie ihm Hinterkopf, dass der Mann eine Melodie pfiff als er davon ging.

Als ob \emph{das} sie einschüchtern würde.

Act 4:

„Entschuldigt, ich bin hier raus,“ sagte Lee Jordan. „Ich bin mehr der Typ für riesige Spinnen.“

Der Junge-der-überlebte hatte gesagt, er hätte \emph{wichtige} Arbeit für den Orden des Chaos, etwas Ernstes und Geheimes, bedeutungsvoller und schwieriger als ihre üblichen Streiche.

Und dann hatte sich Harry zu einer inspirierenden, wenn auch vagen Rede aufgeschwungen. Eine Rede mit der Aussage, Fred und George und Lee hätten beachtliches Potenzial, wenn sie nur lernen würden \emph{bizarrer} zu werden. Um das Leben anderer \emph{surreal} zu machen, statt sie nur mit dem Äquivalent von auf Türen positionierten Wassereimern zu überraschen. (Fred und George hatten interessierte Blicke ausgetäuscht, daran hatten sie nie gedacht.) Harry Potter hatte ein Bild des Streiches, den die auf Neville gespielt hatten, heraufbeschwören \later für den, hatte Harry mit Reue erwähnt, der Sprechende Hut ihn ausgeschimpft hatte \later der aber Neville dazu gebracht hatte \emph{um seinen Verstand zu fürchten}. Für Neville musste es sich angefühlt haben, als ob er in ein Paralleluniversum transportiert worden wäre. Genauso wie es sich alle gefühlt hatten, als sie gesehen hatten wie Snape sich entschuldigte. Das war die \emph{wahre Macht der Streiche.}

\emph{„Seit ihr dabei?“, hatte Harry Potter geschrieen, und Lee Jordan hatte dies verneint.}

\emph{„}Wir sind dabei,“ sagte Fred oder möglicherweise auch George, da kein Zweifel bestand, dass Godric Gryffindor zugestimmt hätte.

\emph{Lee Jordan warf ihnen ein bedauerndes Lächeln zu, stand auf und verließ den verlassenen und Q}uietus-ten Korridor, wo die vier Mitglieder des Orden des Chaos sich getroffen hatten und sich in einem verschwörerischen Kreis zusammengesetzt hatten.

Die drei Mitglieder des Orden des Chaos kamen zur Sache.

\emph{(Es war nicht} so traurig. Fred und George würden weiterhin mit Lee Jordan an riesigen-Spinnen-Streichen arbeiten, so wie immer. Sie hatten zur angefangen es den Orden des Chaos zu nennen, um Harry Potter zu rekrutieren, nachdem Ron ihnen erzählt hatte, Harry wäre seltsam und böse und Fred und George sich entschieden hatte Harry zu retten, indem sie ihm wahre Freundschaft und Zuneigung zeigten. Glücklicherweise schien dies nicht mehr notwenig zu sein \later obwohl sie sich dabei nicht ganz sicher waren…)

\emph{„Also,“ sagte einer der Zwillinge, „}um was geht es?“

„Rita Kimmkorn,“ erwiderte Harry. „Wisst ihr wer das ist?“

Fred und George nickten, stirnrunzelnd.

„Sie hat angefangen Fragen über mich zu stellen.“

Das waren nicht gerade gute Neuigkeiten.

„Könnt ihr erraten, was ich von euch möchte?“

Fred und George sahen einander an, ein bisschen verwirrt. „Du möchtest, dass wir ihr eine unserer spannender Süßigkeiten unterjubeln?“

\emph{„Nein,“ a}ntwortete Harry. „Nein, nein, nein! Das ist riesige-Spinnen-Denken! Kommt schon, was würdet ihr machen, wenn ihr gehört hättet, Rita Kimmkorn wäre auf der Suche nach Gerüchten über euch?“

Das machte es offensichtlich.

Auf Fred und Georges Gesichtern zeichnete sich der Anklang eines Grinsens ab.

„Gerüchte über uns selbst verbreiten,“ antworteten sie.

\emph{„}Exakt,“ sagte Harry, breit grinsend. „Aber es dürfen nicht einfach irgendwelche Gerüchte sein. Ich möchte den Leute beibringen, niemals dem zu glauben, was Zeitungen über Harry Potter sagen, genauso wie die Muggel nichts über Elvis in der Zeitung glauben. Zunächst dachte ich nur daran, Rita Kimmkorn mit so vielen Gerüchten zu überfluten, sodass sie wissen würde, was sie glauben sollte, aber dann würde sie sich nur die Rosinen herauspicken, jene die plausibel und schlecht klangen. Also möchte ich, dass ihr eine Geschichte über mich erfindet und irgendäüwie Rita Kimmkorn dazu bringt sie zu glauben. Es muss aber etwas sein, bei dem jeder danach wissen wird, dass es gefälscht war. Wir planen Rita Kimmkorn und ihre Redakteure zu täuschen, und danach einen Beweis zu liefern, dass alles falsch war. Und natürlich \later da dies die Anforderungen sind \later muss die Geschichte so lächerlich wie überhaupt möglich sein, und dennoch gedruckt werden. Versteht ihr was ich von euch möchte?“

\emph{„Nicht ganz…“ antwortete Fred oder George langsam. „Du möchtest, dass wir die Geschichte} erfinden?

\emph{„}Ich möchte, dass ihr alles macht,“sagte Harry Potter. „Ich bin zurzeit etwas beschäftigt, außerdem möchte ich in der Lage sein wahrheitsgemäß sagen zu können, ich wusste nicht was kommt. Überrascht mich.“

Für einen Moment zeigte sich ein sehr böses Grinsen auf den Gesichtern von Fred und George.

Dann wurden sie wieder ernst. „Aber Harry, wir wissen wirklich nicht, wie man soetwas macht \later“

\emph{„Dann findet es heraus,“ sagte Harry. „Ich habe Vertrauen in euch. Nicht vollstes Vertrauen, aber wenn ihr es nicht tun} könnt, erzählt es mir und ich versuche es mit jemand anderem oder mache es selbst. Falls ihr eine wirklich gute Idee habt \later für beides, die lächerliche Geschichte und wie ihr Rita Kimmkorn und ihre Redakteure davon überzeugen wolltet es auch zu drucken \later dann fangtruhig an und macht es. Aber nehmt nichts Mittelmäßiges. Wenn euch nicht etwas fantastisches einfällt, sagt es nur.

Fred und George tauschten besorgte Blicke.

„Mir fällt nicht ein,“ sagte George.

„Mir auch nicht,“ sagte Fred. „Sorry.“

Harry starrte sie an.

Und dann begann er zu erklären, wie man es schaffte auf Ideen zu kommen.

Es wäre bekannt, dass es mehr als zwei Sekunden in Anspruch nahm, meinte Harry.

Man nenne \emph{nie} \emph{irgendein} Problem unmöglich, sagte Harry, bis man sich nicht eine richtige Uhr genommen hatte und fünf Minuten, bestimmt durch die Bewegung des Minutenzeigers, über das Problem nachgedacht hatte. Nicht fünf metaphorische Minuten, fünf Minuten auf einer reallen Uhr.

Und \emph{außerdem,} sagte Harry, mit eindringlicher Stimme und mit der rechten Hand hart auf den Boden schlagend, begänne man nicht sofort nach Lösungen zu suchen.

Harry setzte zu einer Erklärung von einem Test an, der von Norman Maier durchgeführt wurde, der etwas namens Organisationspsychologe war und der zwei verschiedene Mengen an Gruppen aufgetragen hatte ein Problem anzugehen.

Das Problem, sagte Harry, handelte von drei Arbeitnehmern mit drei Jobs. Die Nachwuchskraft wollte nur den leichtesten Job machen. Der erfahrende Angestellte wollte häufig den Job wechseln, um Langeweile zu vermeiden. Ein Effizienzexperte, hatte vorgeschlagen der Nachwuchskraft den leichsten Job zu geben und dem erfahrenden Angestellten den schwierigsten Job, was 20\% produktiver sein sollte.

\emph{Eine Hälfte} der Problemlösungsgruppen hatte die Anweisung erhalten „Schlag keine Lösungen vor, bis das Problem nicht so vollstängig wie möglich diskutiert worden ist, ohne schon welche vorzuschlagen.“

\emph{Die andere}n Problemlösungsgruppen hatte keine Anweisungen erhalten. Und diese Leute hatten das Natürliche gemacht und auf die Präsenz eines Problems reagiert indem sie Lösungen vorschlugen. Und die Leute waren ihren Losungen gegenüber anhänglich geworden, und hatten angefangen darüber zu streiten, die relative Wichtigkeit von Freiheit versus Effizienz erörternd; und so weiter.

\emph{Die erste Gruppe an Problemlösern, die die Anweisung erhalten hatten, zunächst das Problem zu} diskutieren und danach zu lösen, waren viel häufiger auf die Lösung gestoßen, dem Neuling den einfachsten Job zu lassen und die anderen beiden zwischen den Jobs rotieren zu lassen, für die Daten des Experten eine Verbesserung von 19\% voraussagten.

\emph{Direkt anzufangen Lösungen vorzuschlagen,} brachte alles völlig aus der Ordnung. Als ob man eine Essen mit dem Dessert beginnen würde, nur schlecht.

\emph{(Harry} hatte außerdem jemand namens Robyn Dawes zitiert, der sagte, je schwieriger ein Problem ist, desto häufiger würden die Leute versuchen es direkt zu lösen.)

\emph{Also würde Harry dieses Problem Fred und George überlassen, und sie würden alle Aspekte davon diskutieren und Ideen über alles auch nur entfernt} Relevantes sammeln. Und sie sollten nicht versuchen sich eine tatsächliche Lösung auszudenken, bis sie diesen Teil nicht abgeschlossen hatten, solange sie natürlich noch nicht zufälligerweise auf eine hervorragende Idee gestoßen wären; in diesem Fall sollten sie es für später aufschreiben und dann weiter nachdenken. Und er wolle nichts für mindestens eine Woche von irgendwelchem sogenannten Scheitern, sich etwas auszudenken hören. Einige Menschen verbrachten Jahrzehnte damit sich etwas auszudenken.

„Noch Fragen?“ sagte Harry.

Fred und George starrten sich an.

„Mir fallen keine ein.“

„Mir auch nicht.“

\emph{Harry} hustete leicht. „Ihr habt noch nicht nach eurem Budget gefragt.“

\emph{Budget?, dachten sie.}

\emph{„Ich könnte euch einfach den Betrag nennen,“ meinte Harry. „Aber ich glaube} das wird inspirierender sein.“

Harrys Hände glitten in seinen Umhang, und zogen etwas hervor \later

Fred und George fielen beinahe zu Boden, obwohl sie bereits saßen.

„Geb es nicht aus nur um es auszugeben,“ sagte Harry. Auf dem Steinboden vor ihnen glitzerten eine absolut lächerliche Menge an Geld.

„Gebt es nur aus wenn die Großartigkeit eurer Idee es benötigt; und was die Großartigkeit eurer Idee benötigt, zögert nicht auszugeben. Wenn etwas übrig bleibt, gebt es danach eben zurück, ich vertraue euch. Oh, und ihr bekommt zehn Prozent von dem was hier ist, unabhängig davon wie viel ihr am Ende ausgebt \later“

\emph{„Das} können wir nicht!“ entfuhr es einem der Zwillinge. „Wir akzeptieren kein Geld für diese Art von Sachen!“

\emph{(Die Zwillinge nahmen nie Geld für etwas Illegales an. Ambrosius Flume unbekannt, verkauften sie alle seine Waren mit null Prozent Aufschlag. Fred und George wollten in Lage sein auszusagen \later} unter dem Einfluss von Veritaserum, falls nötig \later sie wären keine profitierenden Kriminellen gewesen, sie hätten nur einen öffenlichen Dienst angeboten.)

\emph{Harry runzelte die Stirn. „Aber ich fordere euch dazu auf hier wirkliche Arbeit hineinzustecken. Ein Erwachsener würde für soetwas bezahlt werden, und es würde} noch immer als ein Gefallen unter Freunden gelten. Man kann hierfür nicht mal eben Menschen anstellen.“

Fred und George schüttelten den Kopf.

„Na gut,“ sagte Harry. „Ich werde ich einfach teure Weihnachtsgeschenke kaufen und wenn ihr versucht sie zurückzugeben werde ich sie verbrennen. Jetzt \emph{wisst} ihr nicht einmal, wie viel ich für euch ausgeben werde, außer natürlich, dass es mehr sein wird, als wenn ihr direkt das Geld genommen hättet. Und ich werde euch diese Geschenke \emph{sowieso} kaufen, also denkt \emph{daran} bevor ihr mir erzählt \emph{euch wäre nichts hervorragendes eingefallen.“}

Harry stand lächelnd auf, und wandte sich ab, während Fred und George noch immer mit offenem Mund staunten. Er marschierte einige Schritte weit weg und kehrte sich dann um.

„Oh, noch etwas,“ sagte Harry. „Lasst Professor Quirrell da raus, was auch immer ihr macht. Er mag keine Publicity. Ich weiß, es wäre einfacher die Leute seltsame Sachen über den Verteidigungslehrer glauben zu machen, als bei jemand anderem, und es tut mir leid euch hier im Weg zu sein, aber bitte, haltet Professor Quirell hier raus.“

Und Harry drehte sich wieder um und machte noch einige Schritte \later

Schaute noch ein letztes Mal zurück, und sagte sanft, „Vielen Dank.“

Und ging.

Es entstand eine lange Pause, nachdem er sie verlassen hatte.

„Also,“ sagte einer.

„Also,“ sagte der andere.

„Der Verteidigungslehrer mag also keine Publicity.“

„Harry kennt uns nicht besonders gut, oder?“

„Nein, tut er nicht.“

„Aber dafür werden wir natürlich sein Geld nicht verwenden.“

„Natürlich nicht, das wäre nicht richtig. Wir werden den Verteidigungslehrer separat machen.“

„Wir werden ein paar Gryffindors dazu kriegen, Kimmkorn etwas zu schreiben, sagen wir…“

„… sein Ärmel war einmal im Verteidigungsunterricht hoch gerutscht und sie hätten das Dunkle Mal gesehen…“

„und dass er Harry Potter wahrscheinlich ganz viele schreckliche Sachen beibringe…“

„… und dass er der schlechteste Verteidigungslehrer ist, an den sich sogar in Hogwarts jemand erinnern kann, er scheitere nicht nur daran uns zu unterrichten, er macht auch alles falsch, das genaue Gegenteil vom eigentlichen Inhalt…“

„… wie damals als er behauptete, man könne den Todesfluch nur mithilfe von Liebe wirken, was ihn im Prinzip nutzlos machte.“

„Der gefällt mir.“

„Danke.“

„Ich wette dem Verteidigungslehrer gefällt es auch.“

„Er hat einen Sinn für Humor. Er hätte uns nicht so genannt, wenn nicht einen Sinn für Humor hätte.“

„Aber werden wir wirklich in der Lage sein Harrys Auftrag zu erfüllen?“

„Harry hat uns angewiesen, das Problem zu diskutieren bevor wir versuchen es zu lösen, also lass uns das machen.“

Die Weasley-Zwillinge entschiedenen, dass George der Enthuastische sein würde, während Fred anzweifelte.

„Es wirkt alles widersprüchlich,“ sagte Fred. „Er möchte es so lächerlich haben, dass jeder Kimmkorn auslacht und weiß, dass es falsch war und er möchte, dass Kimmkorn es glaubt. Wir können nicht beides gleichzeitig machen.“

„Wir werden Hinweise vortäuschen müssen, um Kimmkorn zu überzeugen,“ sagte George.

„War das eine Lösung?“ fragte Fred.

Sie zogen das in Betracht.

„Vielleicht,“ sagte George, „aber ich denke nicht, wir sollten es \emph{so} eng sehen, oder?“

Die Zwillinge zuckten hilflos mit den Achseln.

„Dann müssen die Hinweise gut genug sein, um Kimmkorn zu überzeugen,“ sagte Fred. „Können wir wirklich so etwas selbst machen?“

„Wir müssen nicht alles selbst machen,“ erwiderte George und deutete auf den Goldhaufen. „Wir können Leute anstellen, um uns zu helfen.“

Die Zwillinge sahen gedankenverloren aus.

„Das könnte Harrys Budget ziemlich schnell verbrauchen,“ sagte Fred. „Das ist viel Geld für uns, aber es nicht viel Geld für jemanden wie Flume.“

„Vielleicht werden uns manche Rabatte geben, wenn sie wissen, dass es für Harry ist,“ sagte George. „Doch, am Allerwichtigsten, was wir auch tun, es muss \emph{unmöglich} sein.“

Fred blinzelte. „Was meinst du mit, \emph{unmöglich?}“

\emph{„S}o unmöglich, dass wir keinen Ärger bekommen, weil niemand glaubt wir hätten es tun können. So unmöglich, dass sogar Harry anfängt sich Gedanken zu machen. Es muss surreal sein, es muss die Leute dazu bringen, ihren eigenen Verstand anzuzweifeln, es muss…\emph{besser sein als Harry.“}

Freds Augen waren vor Erstaunen weit geöffnet. Dies kam manchmal zwischen ihnen vor, aber nicht häufig. „Aber warum?“

„Es waren Streiche. Es waren \emph{alles} Streiche. Der Kuchen war ein Streich. Das Erinnermich war ein Streich. Kevin Entwhistles Katze war ein Streich. Snape war ein Streich. \emph{Wir sind} die besten Streichespieler in Hogwarts, werden wir uns überrollen lassen, uns ergeben ohne einen Kampf?“

„Er ist der Junge-der-überlebte,“ sagte Fred.

„Und wir sind die Weasley-Zwillinge! Er \emph{fordert uns heraus}. Er hat gesagt, wir könnten das machen was er macht. Aber ich wette, er glaubt nicht wir würden jemals so gut wie \emph{er} sein.“

„Er hat Recht,“ sagte Fred, ziemlich nervös. Die Weasley-Zwillinge waren \emph{manchmal} anderer Meinung, sogar wenn sie die gleiche Informationen hatten, doch jedes Mal wirkte es unnatürlich, so als ob zumindest einer von ihnen etwas falsch machte. „Das ist \emph{Harry Potter} über den wir hier reden. Er schafft das Unmögliche. Wir nicht.“

„Doch, wir können es,“ sagte George. „Und wir müssen \emph{noch} unmöglicher sein als er.“

\emph{„}Aber \later“ sagte Fred.

„Es ist, was Godric Gryffindor machenwürde,“ sagte George.

Das klärte es und die Zwillinge sprangen wieder zurück in… welcher Zustand auch immer für sie normal war.

„Also gut, dann \later“

„\later lass uns darüber nachdenken.“

